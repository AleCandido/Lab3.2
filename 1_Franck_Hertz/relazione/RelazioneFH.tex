\documentclass[10pt,a4paper]{article}

\usepackage[utf8]{inputenc}
\usepackage[T1]{fontenc}	
\usepackage[italian]{babel}
\usepackage{amsmath}
\usepackage{amsfonts}
\usepackage{amssymb}
\usepackage{graphicx}

\usepackage[left=2cm,right=2cm,top=2cm,bottom=2cm]{geometry}
\geometry{a4paper}

\usepackage{booktabs} % for much better looking tables
\usepackage{verbatim}
\usepackage{subfig} % make it possible to include more than one captioned figure/table in a single 

% pacchetti che mi fanno schifo ma uso lo stesso (Bob è scemo, ma anche Ale...)
\usepackage[cdot, thickqspace, squaren]{SIunits}
% il miglior pacchetto che potessi desiderare
\usepackage{float}
% macro che mi piacciono
\def\code#1{\texttt{#1}}


\title{Esperienza di Franck Hertz}
\author{Gruppo BL \\ Candido Alessandro, Luzio Andrea, Mazziotti Fabrizio}

\begin{document}

\maketitle

\section{Scopo}
Lo scopo di questa esperienza è di verificare la struttura discreta dei livelli energetici degll'atomo di neon e di stimare la sua energia di eccitazione mediante lo studio degli effetti dissipativi negli urti anelastici di elettroni su atomi di neon.

\section{Strumentazione }

\begin{itemize}
 \item Oscilloscopio digitale Tektronix TDS 1012.
 	\begin{itemize}
 		\item Lo strumento è affetto da errore sistematico del 3 \% sulle scale di tensione utilizzate, e di 100 ppm sulle scale di tempo utilizzate.
 	\end{itemize}
 \item Tetrodo a gas neon ELWE U8482230
 \item Sistema di alimentazione e lettura di corrente ELWE

\end{itemize}

\section{Osservazione degli effetti degli urti anelastici elettrone-neon}

Dopo aver eseguito la procedura di accensione suggerita per l'apparato schematizzato in \figurename{~\ref{fig:circuito}},si è osservata la struttura luminosa nel tetrodo a gas al variare delle tensioni $U_{A}$ e $U_{G}$.

\begin{figure}[h!]
	\centering
	\includegraphics[width=0.75\textwidth]{../grafici/schema_apparato.png}
	\caption{Schema dell'apparato utilizzato nell'esperienza.}
	\label{fig:circuito}
\end{figure}

Come ci si aspetta, mantenendo fisso $U_{G}$ e aumentando la tensione anodica si osserva la comparsa di bande luminose nella parte arancione dello spettro.
Viceversa fissata la tensione anodica e diminuendo $U_{G}$ si ottengono risultati simili, ma l'intervallo su cui è possibile far variare $U_{G}$ è più ristretto (vi sono altre differenze, come descritto nell'Osservazione che segue). In effetti ciò che è rilevante al fine di accelerare gli elettroni è la differenza di potenziale $(U_{A}-U_{G})$, così da permettergli di fare urti anelastici con il neon (cioè per far sì che l'energia degli elettroni sia superiore alla differenza di energia tra due livelli energetici).

\paragraph{Osservazione} La presenza della griglia di controllo permette di definire una superficie equipotenziale, in modo che gli elettroni vengano accelerati come se fossero all'interno di un condensatore a facce piane parallele. Si può notare che diminuendo la tensione ai capi di questa griglia fino a 0, ad un certo punto le bande risultano distorte, in particolare diventando più spesse al centro del tetrodo. Questo è dovuto al fatto che in assenza di potenziale sulla griglia di controllo, il campo elettrico non è proprio uniforme nella regione attraversata dagli elettroni, tranne al più nella zona centrale; ci aspettiamo infatti deviazioni dal comportamento ideale (condensatore infinito) nelle zone periferiche (effetti di bordo).

\paragraph{} Si è fissato $U_{G} = \unit{3.8 \pm 0.1}{V}$, $U_{F}=\unit{8.0 \pm 0.1}{V}$, $U_{E} = \unit{10.6 \pm 0.1}{V}$ \footnote{Per quanto riguarda l'errore su tali misure si veda la Sezione~\ref{errELWE}} e si sono cercate le tensioni $U_{A}$ per le quali compare nelle immediate vicinanze della griglia anodica la prima banda luminosa, la seconda, etc (la tensione frenante $U_E$ è irrilevante perché l'individuazione dei massimi è stata effettuata guardando attraverso la finestra di osservazione e non visualizzando la curva $I_{C}$ vs $U_{A}$ sull'oscilloscopio, ed è quindi indipendente dal comportamento degli elettroni oltre la seconda griglia). La comparsa delle bande corrisponde ai valori massimi della corrente $I_{C}$. Poiché le misure sono state effettuate ad occhio nudo, si sono stimati gli errori su di esse valutando la tensione minima e massima per cui si vedeva la comparsa di una banda. I risultati sono riportati in \tablename{~\ref{tab:massimi}}.

\begin{table}[h!]
\centering
\begin{tabular}{c|c|c|c|c}
\hline
Banda &1&2&3&4\\
\hline 
$U_{A} \unit{}{(\volt)}$ & 22.5 $\pm$ 1.5 & 41.0 $\pm$ 1.5 & 54.5 $\pm$ 1.0 & 72.0 $\pm$ 4.0  \\ 
\hline
\end{tabular}
\captionof{table}{Tensioni $U_{A}$ a fissato $U_{G}$ per cui si osservano comparire le varie bande luminose nelle immediate vicinanze della griglia anodica.}
\label{tab:massimi}
\end{table}


\subparagraph{Differenze} In prima approssimazione ci si aspetta che le differenze fra le tensioni $U_A$ relative a due massimi successivi siano compatibili con l'energia dello stato del neon che che gli elettroni vanno ad eccitare. In tabella sono riportate tali differenze (\tablename{~\ref{tab:differenze}}).

\begin{table}[h!]
	\centering
	\begin{tabular}{c|c|c|c}
		\hline
		Bande (j-i) &1-2&2-3&3-4\\
		\hline 
		$U_{A_i}-U_{A_j} \unit{}{(\volt)}$ & 19 $\pm$ 3  & 13.5 $\pm$ 2.5 & 18 $\pm$ 5\\ 
		\hline
	\end{tabular}
	\captionof{table}{Differenze fra le tensioni in \tablename{~\ref{tab:massimi}}}
	\label{tab:differenze}
\end{table}

La compatibilità fra loro non è ottima, ma non merita ulteriore indagine, infatti la definizione di \emph{comparsa di una banda} è abbastanza ambigua relativamente al fenomeno in esame, per cui gli errori considerati sono evidentemente sottostimati.

\begin{figure}[h!]
	\centering
	\includegraphics[width=0.80\textwidth]{../oscilloscopio/CorrenteAnodicaTempo1.png}
	\caption{Curva $I_{C} - t$ con rampa per il potenziale $U_{A}$. I valori dei potenziali sono: $U_{F}=\unit{8.0 \pm 0.1}{V}$, $U_{E}=\unit{9.6 \pm 0.1}{V}$, $U_{G}=\unit{3.9 \pm 0.1}{V}$.}
	\label{andamento}
\end{figure}


\subsection{Visualizzazione curva $I_{C}$ vs $U_{A}$ su oscilloscopio}
Si è dunque attivato il generatore di rampa per $U_{A}$ e si è posto l'oscilloscopio in modalità Y-t per osservare la curva $I_{C}$ vs $U_{A}$ (la stessa curva si può osservare in modalità X-Y, dove X = $U_{G}$/10 e Y $\propto I_{C}$, data la linearità della rampa) e la rampa stessa, facendo attenzione a regolare il guadagno dell'OpAmp in modo da non saturarne l'uscita. L'andamento è mostrato in \figurename{~\ref{andamento}}.

\paragraph{} All'aumentare di $U_{A}$, la corrente $I_{C}$ cresce fino a raggiungere un massimo quando l'energia degli elettroni è in grado di eccitare gli atomi di neon(comparsa della prima banda). La presenza del campo frenante dovuto al potenziale $U_{E}$ non permette a questi elettroni, che hanno perso energia in seguito all'urto, di arrivare all'anodo e quindi la corrente registrata diminuisce. la corrente $I_{C}$ ha un minimo quando la banda luminosa si distacca dalla griglia anodica, segno del fatto che gli elettroni hanno perso troppa energia per eccitare di nuovo il neon.
Aumentando ancora $U_{A}$ la corrente reinizia ad aumentare fino a raggiungere un nuovo massimo(comparsa della seconda banda) e così via.

Agendo sulla manopola del potenziale $U_{E}$ si è osservato il variare della curva $I_{C}$ vs $U_{A}$ al variare di $U_{E}$ ottenendo come risultati curve simili alla \figurename{~\ref{andamento}}, come ad esempio quella mostrata in \figurename{~\ref{task6.3}}.
Ci si può aspettare che ci sia una certa distribuzione sull'energia degli elettroni, sia per quelli che hanno fatto urti elastici, sia per gli altri, dovuta principalmente al processo termoionico. All'aumentare di $U_E$ quelli a minore energia non sono più in grado di raggiungere l'anodo con una conseguente diminuzione della corrente anodica.

In \figurename{\ref{task6.3}} il grafico mostrato è proprio quello della curva $I_{C} - U_{A}$ con valori di $U_E$ e $U_G$ tali per cui i minimi sono pressoché allineati sull'asse $I_C = 0$.

Il valore minimo di $U_A$ non è particolarmente rilevante per la ricerca dei massimi, in quanto c'è una zona abbastanza ampia per basse $U_A$ in cui la curva $I_{C} - U_{A}$ è piuttosto piatta, mentre per la tensione massima della rampa si è potuto scegliere il massimo fornito dall'alimentatore, che consente di osservare il quarto massimo, ma non manda in saturazione l'amplificatore opportunamente regolato (si veda ancora \figurename{\ref{task6.3}}).

\begin{figure}[h!]
	\centering
	\includegraphics[width=0.80\textwidth]{../oscilloscopio/Task6_3.png}
	\caption{Curva $I_{C} - t$ per $U_{E}=\unit{9.7 \pm 0.1}{\volt}$ e .$U_{G}=\unit{3.5 \pm 0.1}{\volt}$}
	\label{task6.3}
\end{figure}


\subsection{Individuazione dei massimi in funzione di $U_E$}

Si è proceduto come descritto sulla scheda: una volta trovato il valore di $U_E$ per cui tutti i massimi si allineavano sullo 0 si è diminuita la tensione $U_E$ gradualmente fino a 0, acquisendo di volta in volta i dati relativi alla curva $I_{C} - U_{A}$ e il valore corrispondente della tensione $U_E$.

Si riportano in \figurename{~\ref{fig:UEex1}-~\ref{fig:UEex2}-~\ref{fig:UEex3}} tre grafici, relativi ad alcune tensioni rappresentative degli andamenti dei massimi.
Si è scelto di procedere con una spaziatura di $\sim \unit{1}{\volt}$ in $U_E$, tranne vicino a 0 in cui si è aumentato il passo a $\sim \unit{1.5}{\volt}$ dato che il comportamento diventava sempre più banalmente monotono.

\begin{figure}[H]
	\centering
	\begin{minipage}{0.49\textwidth}
		\centering
		\includegraphics[width=0.95\textwidth]{../oscilloscopio/Task9_9.png}
		\caption{Curva $I_{C} - U_{A}$, con $U_E = \unit{10.0 \pm 0.1}{\volt}$}
		\label{fig:UEex1}
	\end{minipage}
	\begin{minipage}{0.49\textwidth}
		\centering
		\includegraphics[width=0.95\textwidth]{../oscilloscopio/Task9_5.png}
		\caption{Curva $I_{C} - U_{A}$, con $U_E = \unit{6.0 \pm 0.1}{\volt}$}
		\label{fig:UEex2}
	\end{minipage}
	\begin{minipage}{0.55\textwidth}
		\centering
		\includegraphics[width=0.95\textwidth]{../oscilloscopio/Task9_0.png}
		\caption{Curva $I_{C} - U_{A}$, con $U_E = \unit{0.3 \pm 0.1}{\volt}$}
		\label{fig:UEex3}
	\end{minipage}
\end{figure}

Si riporta quindi la \tablename{~\ref{tab:maxfit}}, contenente i valori dei massimi individuati per i 10 valori di $U_E$ esaminati.

\begin{table}[h!]
	\centering
	\begin{tabular}{c|c|c|c}
		\hline
		$U_E \unit{}{(\volt)}$ & 1 & 2 & 3 \\
		\hline 
		0.3 $\pm$ 0.1 & * & * & * \\
		1.5 $\pm$ 0.1 & 20.065912181026825 $\pm$ 6.e-12 & * & *  \\
		3.0 $\pm$ 0.1 & 18.180105600033844 $\pm$ 1.6e-12 & 35.480132102176825 $\pm$ 1.4e-12 & *  \\
		4.0 $\pm$ 0.1 & 18.287762310420632  $\pm$ 1.5e-12 & 35.56243494429923 $\pm$ 9.6e-13 & *  \\
		5.0 $\pm$ 0.1 & 18.051815714315318  $\pm$ 8.1e-13 & 35.49168824230213 $\pm$ 7.e-13& 68.79959929414346 $\pm$ 1.8e-11 \\
		6.0 $\pm$ 0.1 & 17.84856829877407   $\pm$ 4.4e-13 & 35.6722840495447 $\pm$ 4.7e-13 & 56.19769987201788 $\pm$ 3.4e-13\\
		7.0 $\pm$ 0.1 &18.33203126115184 $\pm$ 5.6e-13 & 35.793169756694 $\pm$ 3.5e-13 & 56.40857843207049 $\pm$ 2.2e-13 \\
		8.0 $\pm$ 0.1 & 18.165680576428752 $\pm$ 7.e-13 & 35.515109603773624$\pm$ 5.2e-13 & 55.76852031379927 $\pm$ 3.5e-13 \\
		9.0 $\pm$ 0.1 & 18.582315564995 $\pm$  4.e-13 & 35.97854084257551 $\pm$ 2.7e-13& 56.56157463003832 $\pm$ 2.2e-13 \\
		10.0 $\pm$ 0.1 & 18.827216231727 $\pm$ 3.2e-13 & 36.376256518495374 $\pm$ 1.8e-13& 56.69422853493564 $\pm$ 2.2e-13 \\
		\hline
	\end{tabular}
	\captionof{table}{Tabella dei valori dei massimi della curva $I_{C} - U_{A}$ in funzione di $U_E$. Si sono indicati con * i massimi assenti}
	\label{tab:maxfit}
\end{table}

Ci si attende che la spaziatura fra due massimi successivi sia pressoché costante \footnote{A meno di effetti dovuti al cammino libero medio, trascurabili in prima approssimazione}.

%Se vuoi qui puoi riportare la tabella delle differenze

Si riporta infine in \figurename{~\ref{fig:final}} il grafico dell'andamento dei massimi di corrente anodica in funzione della tensione frenante $U_E$.

\begin{figure}[h!]
	\centering
%	\includegraphics[width=0.80\textwidth]{../oscilloscopio/}
	\caption{Andamento della tensione $U_A$ corrispondente ai massimi di $I_C$, in funzione di $U_E$}
	\label{fig:final}
\end{figure}

\emph{\LARGE Andre io il grafico non lo vedo, quindi queste osservazioni devi farle necessariamente tu ---> Falle! \\	
E scrivi esplicitamente il confronto con le tensioni trovate ad occhio e riportate in \tablename{~\ref{tab:massimi}} \\
Riporto qui le ultime cose da scrivere che sono richieste dalla scheda così da non dover andarle a cercare:
1) Confrontate le tensioni a cui la corrente IC è massima con le tensioni annotate all'inizio.
2) Analizzare i dati raccolti per determinare i valori di UA a cui IC è massimo E minimo a UE fissato e mostrate mediante un grafico la loro variazione al variare di UE. \\
Ovviamente quando hai fatto cancella questo commento}

\section{Note}

\subsection{Sulla stima degli errori del sistema di lettura di corrente ELWE}
\label{errELWE}
Si è ritenuto che l'errore da assumere su tali misure fosse un errore massimo dovuto all'incertezza di lettura (1 digit), supponendo che l'errore statistico fosse trascurabile rispetto a questo.

Infatti la lettura dei valori era stabile nel tempo, e anche osservando i valori di $U_{A, min}$ e $U_{A, max}$ con l'oscilloscopio (inizio e fine rampa) si ha che le fluttuazioni sono dominate dalla discretizzazione.

\subsection{Sulla stima degli errori sulle tensioni relative ai massimi di corrente anodica}

\emph{\LARGE Scrivi qui come hai deciso alla fine di valutare l'errore \\
 Riguardo al cammino libero medio: se hai fatto qualcosa va bene se no possiamo anche scrivere in due parole che lo trascuriamo. In tal caso io scriverei:
 Si ritiene che gli effetti dovuti al fatto che, raggiunta l’energia di eccitazione, l’elettrone percorre in media un tratto pari al libero cammino medio prima di urtare un atomo, in modo da acquistare un’ulteriore energia cinetica, siano trascurabili in prima approssimazione.}

%TODO: cammino libero medio

\section{Conclusioni}
L'interpretazione dei fatti sperimentali qui osservati è già riportata nella scheda relativa all'esperienza, che è da considerarsi come un allegato a questo documento.

Si può affermare che il comportamento osservato è compatibile con l'eccitazione del primo gruppo di livelli del neon, $\sim \unit{16.7}{e\volt}$ sopra il fondamentale, dovuta agli urti anelastici con gli elettroni accelerati dalla d.d.p sulle due griglie.

\end{document}