\documentclass[a4paper,10pt]{article}
\usepackage[utf8]{inputenc}
\usepackage[T1]{fontenc}
\usepackage[italian]{babel}

\usepackage{amsmath}
\usepackage{amsfonts}
\usepackage{amssymb}
\usepackage{graphicx}

\usepackage[left=2cm,right=2cm,top=2cm,bottom=2cm]{geometry}
\geometry{a4paper}

\usepackage{booktabs}
\usepackage{verbatim}
\usepackage{subfig}
\usepackage[italian, sort, noabbrev, capitalise]{cleveref}
\usepackage[bottom]{footmisc}

\usepackage[cdot, thickqspace, squaren]{SIunits}
\usepackage{float}
\setlength\marginparwidth{40pt}
\setlength\marginparsep{1pt}

% macro
\def\code#1{\texttt{#1}}

\title{Esercitazione 11: Semplici circuiti logici e Multivibratori.}
\author{Gruppo BL \\ Candido Alessandro, Luzio Andrea, Mazziotti Fabrizio}

\begin{document}

\maketitle

\section{Scopo e Strumentazione}
Studio dell'applicazione delle porte NAND nella costruzione di semplici circuiti logici e multivibratori.

La strumentazione è quella solitamente presente sul banco di lavoro, e inoltre si è usato:
\begin{itemize}
	\item 2 circuiti integrati \code{SN7400} Quad-NAND Gate;
	\item 1 DIP Switch a 4 interuttori;
	\item 1 diodo \code{1N4148};
	\item 2 diodi LED;
	\item Impulsatore, realizzato nella precedente esperienza;
\end{itemize}

\section{Costruzione di circuiti logici elementari}
Si sono connessi due interruttori tra i due ingressi di una porta NAND, alimentata a $\unit{4.94 \pm 0.03}{\volt}$, e massa e si è posto un diodo LED, con una resistenza di protezione di $\sim \unit{330}{\ohm}$, all'uscita della porta.
Si è verificata la tabella di verità del NAND provando tutte le combinazioni delle posizioni dei due interruttori, verificando lo stato dell'uscita mediante l'accensione del LED.

Si è applicato lo stesso metodo anche gli altri circuiti descritti in questa sezione, e si è verificata la tabella di verità nei vari casi.
Inoltre si è usato l'impulsatore per visualizzare il risultato sull'oscilloscopio; si riportano i grafici nelle \cref{fig:AND,fig:OR,fig:XOR,fig:ADDER}.

\paragraph{AND} Si è realizzato il circuito AND con 2 porte NAND.

\begin{figure}[H]
	\centering
	\begin{minipage}{0.49\textwidth}
		\centering
		\includegraphics[width=\textwidth]{../grafici/AND1.png}
	\end{minipage}
	\begin{minipage}{0.49\textwidth}
		\centering
		\includegraphics[width=\textwidth]{../grafici/ANDard.pdf}
	\end{minipage}
	\caption{Schema del circuito AND e visualizzazione segnali in I/O}
	\label{fig:AND}
\end{figure}

\paragraph{OR} Si è realizzato il circuito OR con 3 porte NAND.

\begin{figure}[H]
	\centering
	\begin{minipage}{0.49\textwidth}
		\centering
		\includegraphics[width=\textwidth]{../grafici/OR1.png}
	\end{minipage}
	\begin{minipage}{0.49\textwidth}
		\centering
		\includegraphics[width=\textwidth]{../grafici/ORard.pdf}
	\end{minipage}
	\caption{Schema del circuito OR e visualizzazione segnali in I/O}
	\label{fig:OR}
\end{figure}

\paragraph{XOR} Si è realizzato il circuito XOR con 4 porte NAND.

\begin{figure}[H]
	\centering
	\begin{minipage}{0.49\textwidth}
		\centering
		\includegraphics[width=\textwidth]{../grafici/XOR1.png}
	\end{minipage}
	\begin{minipage}{0.49\textwidth}
		\centering
		\includegraphics[width=\textwidth]{../grafici/XORard.pdf}
	\end{minipage}
	\caption{Schema del circuito XOR e visualizzazione segnali in I/O}
	\label{fig:XOR}
\end{figure}


\paragraph{Sommatore ad un bit} Si è realizzato il sommatore ad un bit con 5 porte NAND.

\begin{figure}[H]
	\centering
	\begin{minipage}{0.49\textwidth}
		\centering
		\includegraphics[width=\textwidth]{../grafici/Sommatore1.png}
	\end{minipage}
	\begin{minipage}{0.49\textwidth}
		\centering
		\includegraphics[width=\textwidth]{../grafici/ADDERard.pdf}
	\end{minipage}
	\caption{Schema del sommatore ad un bit e visualizzazione segnali in I/O}
	\label{fig:ADDER}
\end{figure}


\subsection{Osservazione} 

Negli schemi elettrici mostrati nelle \cref{fig:AND,fig:OR,fig:XOR,fig:ADDER} si può notare che per alcune porte NAND si è lasciato un ingresso flottante, quando si voleva usarle come porte NOT. Si è sfruttato infatti il fatto che gli ingressi flottanti delle porte corrispondono ad un valore \code{HIGH}, secondo la logica TTL.
Tenendo conto di questo si ottiene infatti (\code{HIGH-HIGH} $\rightarrow$ \code{LOW}) e (\code{LOW-HIGH} $\rightarrow$ \code{HIGH}), cioè il comportamento richiesto per una porta NOT.

Si sarebbe potuto connettere gli ingressi insieme, così da avere in uscita ancora una volta un NOT, infatti (\code{HIGH-HIGH} $\rightarrow$ \code{LOW}) e (\code{LOW-LOW} $\rightarrow$ \code{HIGH}) secondo la tabella di verità di una porta NAND, ed essere indipendenti dalla logica usata; si è preferito fare così solo per risparmiare qualche connessione sulla basetta e avere un po' più di ordine (fisico).

Si è applicato lo stesso metodo anche nei circuito successivi, anche se gli schemi dei circuiti mostrano gli input connessi insieme perché tratti dalla scheda (vedi \cref{fig:MONO,fig:AST,fig:SQGEN}).


\section{Multivibratore MONOSTABILE}
Si è montato il circuito mostrato in \cref{fig:MONO} che rappresenta un Multivibratore Monostabile. 


\begin{figure}[H]
	\centering
	\includegraphics[width=0.9\textwidth]{../grafici/Monostabile.png}
	\caption{Schema del circuito del multivibratore monostabile}
	\label{fig:MONO}
\end{figure}


Si è collegato l'ingresso del multivibratore al generatore di onde, in particolare si è inviata un'onda quadra al suo ingresso con un'ampiezza compresa tra ($88\pm5$)mV e ($4.6\pm0.2$)V , regolando il periodo dell'onda a $(200\pm 1) \mu s$, ossia ad una frequenza di (5000$\pm$25) Hz e il duty cycle (tempo in cui l'onda era a tensione alta rispetto al periodo complessivo) al ($6.7\pm0.3$)\%,. In altre parole si è utilizzato un impulso di $(13.4\pm0.4) \mu s$ ripetuto ogni $(200\pm1) \mu s$.

I componenti impiegati sono stati misurati con il tester digitale, e sono:

\begin{table}[H]
	\centering
	\begin{tabular}{cc}
		$R_1 = \unit{475 \pm5}{\ohm}$ & $C_1 = \unit{108 \pm 5}{\nano\farad}$\\
	\end{tabular}
\end{table}




\paragraph{Funzionamento del circuito}
Si analizzano i grafici mostrati in \cref{fig:OSC,fig:VC} che mostrano rispettivamente le forme d'onda (a regime) in IN-M e OUT-M e le forme d'onda in IN-M e VC, come definiti in \cref{fig:MONO}.

\begin{figure}[H]
	\centering
	\includegraphics[width=0.58\textwidth]{../grafici/monostabileOSC.png}
	\caption{Forme d'onda visualizzate tramite l'oscilloscopio in ingresso (azzurra) e in uscita (gialla) al multivibratore monostabile.}
	\label{fig:OSC}
\end{figure}

Quando l'onda quadra in ingresso è a $\sim 0V$ (\code{LOW}), all'uscita del NAND1 si è nello stato \code{HIGH}. Supponendo che in partenza  all'uscita del circuito si abbia lo stato \code{LOW}, quindi all'uscita del NAND3 lo stato \code{HIGH}, si ottiene che l'uscita del NAND2 è \code{LOW}\footnote{Lo stato del circuito non sarebbe altrimenti stabile, come descritto sotto, e si transirebbe verso lo stato qui descritto.}. Il condensatore, inizialmente scarico, non si carica e all'ingresso del NAND3 si ha lo stato \code{LOW-LOW}. In questa situazione, finché in ingresso si è nello stato \code{LOW}, il circuito è stabile.

A questo punto l'ingresso passa a \code{HIGH}, quindi l'uscita del NAND2 è \code{HIGH}. Il condensatore inizia a caricarsi attraverso la resistenza $R_1$ e quindi in VC si vede prima il passaggio ad un potenziale\footnote{La carica di un condensatore non è istantanea, quindi quando il NAND2 passa a \code{HIGH},inizialmente entrambe le armature sono ancora allo stesso potenziale, cioè alto.} $V_{max} = 2.93\pm 0.02 V$, e poi la carica di esso, che si manifesta attraverso una discesa esponenziale poiché l'armatura di destra del condensatore si deve caricare negativamente (\cref{fig:VC}).

\begin{figure}[H]
	\centering
	\includegraphics[width=0.6\textwidth]{../grafici/monostabileVC.png}
	\caption{Forme d'onda visualizzate tramite l'oscilloscopio in ingresso (azzurra) e in VC (vedi \cref{fig:MONO}) del multivibratore monostabile.}
	\label{fig:VC}
\end{figure}

Adesso l'ingresso non gioca più alcun ruolo: finché la tensione VC non scende al di sotto di $V_{IH}$ del NAND3, si vedrà la discesa esponenziale dovuta alla carica del condensatore, e quindi la durata dell'impulso (\code{HIGH}) in OUT-M dipende solamente da questa carica, cioè è proporzionale al tempo caratteristico di carica del condensatore $\tau = R_1 C_1 = 51 \pm 3 \mu s$, (la costante di proporzionalità trovata vale 0.87 $\pm$ 0.08) e non dipende dalla durata dell'impulso in IN-M. Questa cosa si è anche verificata modificando l'impulso in IN-M. Il valore di tensione per cui il NAND3 commuta è $V_{com} = 1.44\pm 0.04 V$.

Quando il NAND3 commuta (passa da \code{HIGH-HIGH} a \code{LOW-LOW} al suo ingresso, e quindi in uscita a \code{HIGH}), in seguito al valore scelto per il duty cycle in ingresso al circuito, all'ingresso del NAND1 si è nello stato \code{LOW}, quindi l'uscita del NAND2 è \code{LOW}. In OUT-M il segnale passa quindi da \code{HIGH} a \code{LOW}. Il diodo entra in conduzione (resistenza trascurabile) e l'armatura di destra del condensatore non passa da (\code{HIGH} $-~exp(t/\tau)$) a (\code{LOW} $-~exp(t/\tau)$), ma direttamente ad un valore di potenziale più elevato per cui il diodo non è più in conduzione, il tutto in tempi non apprezzabili sulla scala scelta visualizzata sull'oscilloscopio in \cref{fig:VC}. Il valore del minimo del potenziale\footnote{Tutte le misure dei valori per cui il NAND3 commuta, dei massimi e dei minimi dei potenziali visualizzati in VC, sono calcolate facendo il valor medio e la deviazione standard tra le tre misure trovate analizzando la forma d'onda in \cref{fig:VC}.} è $V_{min} = -0.88\pm 0.04 V$.
% non so che valore guardare sul datasheet, ma ce ne possiamo anche fregare

A questo punto il diodo esce dal regime di conduzione e passa a interdizione; in questo passaggio il condensatore si carica attraverso la resistenza $R_1$ passando da un valore negativo di tensione fino ad una tensione di $\sim 0V$. Quindi alla fine il condensatore risulta scarico e si è tornati nella situazione stabile. Non appena in ingresso al circuito si passa da \code{LOW} a \code{HIGH} il ciclo ricomincia.

La funzione del diodo è quindi di non permettere al potenziale in ingresso del NAND3 di essere troppo negativo, evitando così il malfunzionamento di tale porta.


\subsection{Linearità dell'impulso in uscita al circuito}
Si è verificato che l'andamento dell'impulso in OUT-M in funzione della resistenza $R_1$ fosse lineare, come atteso in seguito alle osservazioni precedenti.
Si sono provati altri $4$ valori per la resistenza $R_1$, oltre a quello impiegato già precedentemente, e si è eseguito un fit lineare con tutti e $5$ i valori. Si riportano dunque il grafico del fit e la tabella con le misure, rispettivamente \cref{fig:MONOfit,tab:MONOfit}.

\begin{figure}[H]
	\centering
	\begin{minipage}{0.49\textwidth}
		\centering
		\includegraphics[width=\textwidth]{../grafici/FITmonostabile.pdf}
		\caption{Grafico dell'impulso in OUT-M in funzione della resistenza $R_1$}
		\label{fig:MONOfit}
	\end{minipage}
	\begin{minipage}{0.49\textwidth}
		\centering
		\resizebox{0.7\textwidth}{!}{
			\input{../tabelle/tab_monostabileRes.txt}}
		\captionof{table}{Resistenze utilizzate per verificare la linearità dell'impulso in OUT-M in funzione della resistenza $R_2$.}
		\label{tab:MONOfit}
	\end{minipage}
\end{figure}

Si riportano di seguito i parametri di fit:

\begin{table}[H]
	\centering
	\begin{tabular}{cccc}
		$\chi^2/ndof = 0.64/3$ & $m = \unit{ 0.110\pm0.006 }{\micro\second / \Omega}$ & $q = \unit{ 7\pm3 }{\micro\second}$ & $corr_{mq} = -0.97 $\\
	\end{tabular}
\end{table}

Dove si è indicato con $m$ il coefficiente angolare, con $q$ l'intercetta e con $corr_{mq}$ il coefficiente di correlazione, che risulta prossimo a ~$-1$ come atteso per una retta. I risultati del fit sono in ottimo accordo con quanto atteso.
% l'errore sui tempi è piccolo perchè mme prende solo la tacca dell'oscilloscopio che è palesemente una sottostima. Ci ho aggiunto il 3% di incertezza sistematica sui tempi, ed esce quello che c'è scritto in tabella tranne il chi2 che esce 18. Cosa metto?

\pagebreak[2]
\section{Multivibratore ASTABILE}

Si è montato il circuito in \cref{fig:AST}.

\begin{figure}[H]
	\centering
	\includegraphics[width=0.7\textwidth]{../grafici/Astabile.png}
	\caption{Schema del circuito del multivibratore astabile}
	\label{fig:AST}
\end{figure}

I componenti impiegati sono stati misurati con il tester digitale, e sono:

\begin{table}[H]
	\centering
	\begin{tabular}{cc}
		$R_2 = \unit{990 \pm 10}{\ohm}$ & $C_2 = \unit{109 \pm 5}{\nano\farad}$\\
	\end{tabular}
\end{table}

Si è verificato che in uscita, OUT-A, la forma d'onda fosse un'onda quadra; si sono misurati il periodo, $\Delta T$, e il duty-cycle, $\delta$, dell'onda, ottenendo:

\begin{table}[H]
	\centering
	\begin{tabular}{cc}
		$\Delta T = \unit{210 \pm 1}{\micro\second}$ (4760$\pm$25 Hz) & $\delta = \unit{72 \pm 4}{\micro\second}$ (34$\pm$2\%)\\
	\end{tabular}
\end{table}

Si riporta quindi la forma d'onda su OUT-A (azzurra) in \cref{fig:ASTosc}, assieme alla forma d'onda presente su VC$2$ (gialla).

\begin{figure}[H]
	\centering
	\includegraphics[width=0.7\textwidth]{../grafici/astabileOSC.png}
	\caption{Forme d'onda visualizzate tramite l'oscilloscopio nei punti OUT-A  (azzurra) e VC2 (gialla) della \cref{fig:AST}}
	\label{fig:ASTosc}
\end{figure}

La forma d'onda presente su VC$2$ è il risultato delle curve esponenziali di carica e scarica del condensatore (in realtà sono sempre cariche, perché non si scarica mai liberamente senza una tensione esterna ai capi).

\paragraph{Interpretazione} Per spiegare il comportamento del circuito si deve considerare che la porta NAND$7$ forza la sua uscita ad assumere un valore che sia ben definito, come \code{HIGH} o come \code{LOW}, e altrettanto fa la porta NAND$6$: in questo modo il ramo $R_2-C_2$ ha sempre una differenza di potenziale approssimativamente fissata ai suoi capi, di cui l'unica cosa che varia sostanzialmente è il segno.

Il condensatore si carica dunque, modificando in modo continuo il valore di tensione sul NAND$5$, e raggiunta una data soglia (che risulta essere la stessa in salita in discesa, vedi \cref{fig:ASTosc}), fa commutare NAND5, che a cascata porta tutti gli altri NAND a commutare (sono collegati a catena, l'input di uno all'output del precedente, vedi \cref{fig:AST}).

Il tempo di carica del condensatore è determinato dai valori di $R_2$ e $C_2$, e dalla soglia a cui commuta NAND$5$\footnote{per tempi troppo brevi diventano rilevanti i ritardi delle porte, ma a quel punto non è più possibile considerare come simultanea la commutazione complessiva descritta prima e il circuito può smettere di comportarsi come atteso nel caso descritto}.

Si nota inoltre che all'istante in cui tutte le porte NAND commutano, anche la tensione su VC$2$ salta: infatti il condensatore, che non ha il tempo di scaricarsi, genera la stessa differenza di potenziale tra l'output del NAND$6$ e l'input del NAND$5$ (cioè VC$2$) che generava prima della commutazione, cambiando però istantaneamente il valore dell'output del NAND$6$ cambia pure istantaneamente il valore di VC$2$.

\subsection{Linearità del periodo}
Si è quindi verificato che l'andamento del periodo in funzione della resistenza $R_2$ fosse lineare, come atteso: infatti l'andamento del tempo caratteristico del ramo $R_2-C_2$ è lineare nel valore di $R_2$, ed il periodo è determinato linearmente da tale tempo caratteristico (se il condensatore impiega il doppio del tempo per raggiungere la soglia il periodo sarà doppio).

Si sono provati altri $4$ valori per la resistenza $R_2$, oltre a quello impiegato già precedentemente, e si è eseguito un fit lineare con tutti e $5$ i valori. Si riportano dunque il grafico del fit e la tabella con le misure, rispettivamente \cref{fig:ASTfit,tab:ASTfit}.

\begin{figure}[H]
	\centering
	\begin{minipage}{0.49\textwidth}
		\centering
		\includegraphics[width=\textwidth]{../grafici/FITastabile.pdf}
		\caption{Grafico del periodo in funzione della resistenza $R_2$}
		\label{fig:ASTfit}
	\end{minipage}
	\begin{minipage}{0.49\textwidth}
		\centering
		\resizebox{0.7\textwidth}{!}{
			\input{../tabelle/tab_astabileRes.txt}}
		\captionof{table}{Resistenze utilizzate per verificare la linearità del periodo in funzione di $R_2$.}
		\label{tab:ASTfit}
	\end{minipage}
\end{figure}

Si verifica che la linearità è ottima, e si riportano di seguito i parametri di fit:

\begin{table}[H]
	\centering
	\begin{tabular}{cccc}
		$\chi^2/ndof = 1.60/3$ & $m = \unit{196 \pm 4}{\micro\second / \kilo\ohm}$ & $q = \unit{16 \pm 4}{\micro\second}$ & $corr_{mq} = -0.97$\\
	\end{tabular}
\end{table}

Dove si è indicato con $m$ il coefficiente angolare, con $q$ l'intercetta e con $corr_{mq}$ il coefficiente di correlazione, che risulta prossimo a ~$-1$ come atteso per una retta.

\section{Generatore di onda quadra}

\begin{figure}[H]
	\centering
	\includegraphics[width=0.8\textwidth]{../grafici/SqGen.png}
	\caption{Schema del circuito del generatore di onda quadra}
	\label{fig:SQGEN}
\end{figure}

\subsection{Forma d'onda in uscita dal derivatore}

In questa parte dell'esperienza si mettono in serie i due circuiti per fare un generatore di onde quadre a duty-cycle variabile, collegando monostabile e astabile attraverso un filtro passa-alto (un derivatore) con frequenza di taglio $1/\tau=1/RC=\unit{97 \pm 4}{\kilo\hertz}$.\\
Si è osservato l'output della sezione di alimentazione, a valle del derivatore, e l'output del intero circuito. Come è ben visibile dal oscilloscopio \cref{fig:InMOutM} il multivibratore monostabile è triggerato solo dal fronte d'onda positivo. 


% io metterei una motivazione del perchè è triggerato solo sul fronte d'onda positivo.


\begin{figure}[H]
	\centering
	\includegraphics[width=0.8\textwidth]{../grafici/4InMOutM.png}
	\caption{In blu l'output del multistabile (OUT-M), in arancione l'output del derivatore (IN-M)}
	\label{fig:InMOutM}
\end{figure}

\pagebreak[1]
\subsection{Forma d'onda all'uscita complessiva (OUT-M)}

Ci si aspetta che i due tempi caratteristici del circuito, il periodo e $t_{duty}$\footnote{tempo durante il quale il circuito è \code{HIGH} in un periodo}, siano regolati indipendentemente dalle resistenze $R_2$ ed $R_1$. Questo è logico pensando il circuito come un generatore di forme d'onda quadre (l'astabile) che pilota il monostabile. Il pilotaggio avviene tramite il derivatore che trasforma l'onda quadra in un treno di impulsi. Ne risulta che il periodo dovrebbe essere il periodo dell'astabile, mentre il $t_{duty}$ dovrebbe essere il tempo caratteristico del monostabile. Tutto questo dovrebbe funzionare finché ci si limita a $t_{duty}$ alti rispetto al tempo caratteristico del derivatore ovvero $\unit{10.27 \pm 0.45}{\micro\second}$\footnote{Ci si è tenuti sempre sopra i $\unit{25}{\micro\second}$ dunque si esclude la possibilità che il transiente dato dal derivatore non fosse molto soppresso, dunque irrilevante}.\\
Si è proceduto con la verifica di queste affermazioni. Per prima cosa si è fittato, per diversi valori della resistenza $R_2$, la dipendenza del duty-cycle da $R_1$ aspettandoci un andamento lineare (in effetti si è svolto un fit affine $t_{duty}=mR_1+q$, anche se di fatto ci si aspetta una dipendenza lineare). I risultati di tali fit sono riassunti in \cref{tab:DutyFit} e nel \cref{fig:FitsRighe}. 

\begin{table}[H]
\centering
\begin{tabular}{c|c|c|c} 
$R_2$ & $m [\second/\ohm]$ & $q [\ohm]$ & $\chi^2$\\
\hline
$673 \pm 6$ & $(9.15 \pm 0.22) e-08$ & $(-1.9 \pm 1.3) e-06$ & $3.9$\\
$986\pm 9$ & $(1.102 \pm 0.029) e-07$ & $(-8.2\pm 1.6) e-06$ & $0.5$\\
$1459 \pm 13$ & $(1.212\pm 0.023) e-07$ & $(-1.16 \pm 0.13) e-05$ & $1.0$\\
$820 \pm 8$ & $(1.005 \pm 0.023) e-07$ & $(-4.4\pm 1.3) e-06$ & $1.4$\\
$1174 \pm 10$ & $(1.150 \pm 0.022) e-07$ & $(-1.11 \pm 0.13) e-05$ & 2.1\\
\end{tabular}
\caption{Diversi fit del duty-cicle in funzione della resistenza $R_1$ per diversi valori di $R_2$}
\label{tab:DutyFit}
\end{table}

\begin{figure}[H]
	\centering
	\includegraphics[width=0.8\textwidth]{../grafici/FitsRighe.pdf}
	\caption{Diversi fit del duty-cycle in funzione della resistenza $R_1$ per diversi valori di $R_2$}
	\label{fig:FitsRighe}
\end{figure}

Si sono poi presi i diversi valori dei coefficienti angolari $m$ e se ne è trovata la media e la deviazione standard. \'E stato riportato anche il valore del $\chi^2$ sia per $m$ che per $q$.


\begin{table}[H]
\centering
\begin{tabular}{c|c} 
%qunatità & valore\\
%\hline
$m$&$\unit{0.1076 \pm 0.0010}{\micro\second/\ohm}$ \\
$q$&$\unit{-7.4 \pm 0.6}{\micro\second}$ \\
$\chi^2_m$&$109.3$\\
$\chi^2_q$&$41.3$\\
$dof$&$3$\\
\end{tabular}
\end{table}


I $\chi^2$ ottenuti sono molto alti, totalmente incompatibili con il numero dei gradi di libertà(4), sia per $q$ che per $m$. Ci si può chiedere se ciò sia dovuto alla sottostima dell'errore sugli $m$ e sui $q$ fittati o se il fit rigetti l'ipotesi. 
Si sono plottati quindi gli scarti normalizzati, visibili in  \cref{fig:FitRighe}.
Come si può vedere da quest'ultimo grafico c'è una deriva sistematica di entrambi i parametri in base alla resistenza $R_2$. Questo permette di concludere che l'indipendenza di $t_{duty}$ dalla resistenza dell'astabile è solo approssimativamente rispettata e, con gli strumenti a nostra disposizione si può notare questo scostamento. 

%Si sono anche plottati residui degli scarti in \cref{fig:FitRighe}. \`E chiaramente visibile un andamento sistematico. In effetti..., che con il $\chi^2$ \underline{improbabile} mostrato sopra mostra che il duty-cycle è \underline{solamente approssimativamente}.


\begin{figure}[H]
	\centering
	\includegraphics[width=0.8\textwidth]{../grafici/FitRighe.pdf}
	\caption{Scarti normalizzati di m dalla media e scarti normalizzati di q dalla media}
	\label{fig:FitRighe}
\end{figure}




Si è svolta la stessa procedura per verificare la linearità del periodo rispetto a $R_2$ e la sua indipendenza da $R_1$. Sono stati effettuati i con diversi valori di $R_1$ e sono stati riassunti in \cref{tab:FitsColonne} e nel \cref{fig:FitsColonne}.

\begin{table}[H]
\centering
\begin{tabular}{c|c|c|c} 
$R_1$ & $m [\second/\ohm]$ & $q [\ohm]$ & $\chi^2$\\
\hline
$386 \pm 4$ & $(1.973 \pm 0.035)e-07$ & $(1.55 \pm 0.34)e-05 $& 0.49\\
$820 \pm 8$ & $(1.99  \pm 0.07)e-07 $&  $(1.4  \pm 0.8)e-05 $& 0.23\\
$327 \pm 4$ & $(1.973 \pm 0.035)e-07 $& $(1.55 \pm 0.34)e-05 $& 0.49\\
$986 \pm 9$ & $(1.97  \pm  0.04)e-07 $& $(1.55 \pm 0.34)e-05 $& 0.49\\
$475 \pm 5$ & $(1.96  \pm 0.05)e-07 $&  $(1.7  \pm 0.5)e-05 $& 0.34\\
$674 \pm 6$ & $(1.96  \pm 0.05)e-07 $&  $(1.6  \pm 0.5)e-05 $& 0.09\\
$558 \pm 5$ & $(1.973 \pm 0.035)e-07 $& $(1.55 \pm 0.34)e-05 $& 0.49\\
\end{tabular}
\label{tab:FitsColonne}
\end{table}

\begin{figure}[H]
	\centering
	\includegraphics[width=0.8\textwidth]{../grafici/FitsColonne.pdf}
	\caption{Diversi fit del duty-cycle in funzione della resistenza $R_2$ per diversi valori di $R_1$}
	\label{fig:FitsColonne}
\end{figure}


Questa volta $m$ e $q$ sono indipendenti dalla resistenza $R_1$ (si vede sia dalla tabella dei risultati che dal grafico, dove le rette con diverso $R_1$ si sovrappongono), cioè si ottiene proprio quanto atteso.

\begin{table}[H]
\centering
\begin{tabular}{c|c} 
%qunatità & valore\\
%\hline
$m$ & $\unit{0.1971 \pm 0.0016}{\micro\second/\ohm} $\\
$q$ & ${15.6 \pm 1.5}{\micro\second}$\\
$\chi^2_m$ & $0.15$\\
$\chi^2_q$ & $0.12$\\
$dof$ & $7$\\
\end{tabular}
\label{tab:resColonna}
\end{table}



Il valore dei $\chi^2$ mostra che gli errori mostrati in \cref{tab:FitsColonne} sono probabilmente sovrastimati. Comunque si può concludere che il periodo è dato da una retta affine rispetto a $R_2$ e che nulla dipende da $R_1$. Questo è ancora più evidente dal plot degli scarti normalizzati (\cref{fig:FitColonne}), che non mostra alcun segno di andamento sistematico.

 
\begin{figure}[H]
	\centering
	\includegraphics[width=0.8\textwidth]{../grafici/FitColonne.pdf}
	\caption{Scarti normalizzati di m dalla media e scarti normalizzati di q dalla media}
	\label{fig:FitColonne}
\end{figure}

\subsection{Modulazione periodo e duty-cycle}

In questa sezione dell'esperienza si provato a trovare dei valori di resistenze $R_1$ e $R_2$ tali da avere un periodo di circa $\unit{100}{\micro\second}$ e un duty-cycle del 30 \%. Si è ottenuto il risultato voluto con $R_1=\unit{386 \pm 5}{\ohm}$ e  $R_1=\unit{434 \pm 5}{\ohm}$. In \cref{fig:410030} il risultato visto all'oscilloscopio. 


 \begin{figure}[H]
	\centering
	\includegraphics[width=0.8\textwidth]{../grafici/410030.png}
	\caption{In blu l'output del multistabile (OUT-M), in arancione l'output del derivatore (IN-M)}
	\label{fig:410030}
\end{figure}

Come si può osservare anche dal grafico il periodo è di $\unit{100.0 \pm 2.5}{\micro\second}$ e il duty-cycle di $\unit{30.0 \pm 2.5}{\micro\second}$ (questo, supponendo che gli errori sui tempi siano scorrelati, porta a un duty-cycle di $30 \pm 2.6 \% $, con una probabile sovrastima dell'errore).



\end{document}