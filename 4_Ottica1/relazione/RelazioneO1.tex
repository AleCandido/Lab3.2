\documentclass[a4paper,10pt]{article}
\usepackage[utf8]{inputenc}
\usepackage{graphicx}
\usepackage{verbatim}
%opening
\title{Esperienza di Ottica 1}
\author{Gruppo BL \\ Candido Alessandro, Luzio Andrea, Mazziotti Fabrizio}

\begin{document}

\maketitle

%\begin{abstract}
%Lo scopo di questa esperienza è di
%\end{abstract}

\section{Scopo}
L'esperienza è divisa in due parti:
\begin{itemize}
	\item nella prima parte si vuole determinare la lunghezza d’onda di una riga spettrale emessa dal sodio (la riga gialla);
	\item nella seconda parte ci sono due obiettivi:
	\begin{itemize}
		\item il primo è di valutare la risoluzione dello strumento di misura, uno spettroscopio a reticolo, nella misura delle linee emesse da lampade spettrali;
		\item il secondo è quello di determinare la costante di Rydberg dalla misura della lunghezza d’onda delle righe di emissione dell’idrogeno.
	\end{itemize}
\end{itemize}

\section{Esperienza A: Misura della lunghezza d'onda della riga gialla del sodio}

\subsection{Strumentazione}

\begin{itemize}
	\item spettroscopio a prisma:
	\begin{itemize} %questa lista è solo un abbozzo, devo scriverla bene
		\item due telescopi;
		\item base rotante con goniometro con nonio etc. etc.;
		\item prisma con supporto.
	\end{itemize}
	\item lampada al cadmio;
	\item lampada al sodio;
	\item lente d'ingrandimento, per la lettura del nonio;
	\item torcia.
\end{itemize}

\section{Esperienza B: Misura della costante di Rydberg}

\subsection{Strumentazione}

\begin{itemize}
	\item spettroscopio a reticolo di diffrazione:
%stesse cose di sopra da aggiungere

	\item lampada al mercurio;
	\item lampada al idrogeno;
	\item lampada al sodio;
	\item lente d'ingrandimento, per la lettura del nonio;
	\item torcia.
\end{itemize}


\subsection{Regolazione della geometria dei telescopi}
Lo schema dell'apparato è rappresentato in figura \figurename{~\ref{fig:reticolo}}.
In primo luogo si è posizionata la lampada al mercurio vicino al telescopio fisso e si è ruotato il telescopio di osservazione fino ad osservare l'ordine zero di diffrazione. Si è fissato questo telescopio e si è regolato lo spessore della fenditura, la distanza della lampada da essa e la messa a fuoco del telescopio per visualizzare la riga con la maggior intensità possibile e con uno spessore né troppo piccolo da non permettere la visualizzazione della riga, né troppo grande per evitare di avere un'incertezza grande sulla effettiva posizione della riga rispetto alle distanze in gioco\footnote{La distanza tra i due bordi della riga dell'ordine zero (e poi anche di quelle di ordine superiore), nelle condizioni scelte nell'esperienza, è stimata essere minore di ???(io direi 0.5') che è da confrontare con la scala graduata del nonio, la cui risoluzione massima è di 0.5'.}.
%da completare
La distanza della lampada dalla fenditura non deve essere né troppo grande ($\sim 1 cm$) perché ne risentirebbe la luminosità delle riga, né troppo piccola perché si vedrebbero aloni intorno ad essa.
Si sono poi osservate le righe del primo ordine di diffrazione (quelle più intense,cioè, in ordine per angolo di incidenza decrescente: viola, verde, rosso e un doppietto giallo) e si è verificato che, con gli stessi accorgimenti fatti per la riga dell'ordine zero, esse siano ben messe a fuoco.

\subsection{Calibrazione dello strumento e misura del passo reticolare}
Con la stessa procedura effettuata nella parte A si è effettuata la calibrazione del nonio ottenendo $\alpha_0 = $\footnote{La trattazione dei vari dati acquisiti in questa parte dell'esperienza e dei relativi errori sono trattati nell'apposita sezione 'errori sistematici'.}. Presa questa misura, tutte le letture successive fatte sul nonio e riportate qui saranno riferite a questo valore.
\paragraph{Misura del passo reticolare}

\subsection{Righe di emissione dell'idrogeno}


\subsection{Misura del doppietto del sodio}



%io metterei qui tutta la spiegazione del fatto che abbiamo fatto ciascuno la nostra misura ecc in modo da mettere in tutte le altre parti solo il risultato finale.
\subsection{Errori sistematici}

\end{document}