\documentclass[a4paper,10pt]{article}
\usepackage[utf8]{inputenc}
\usepackage[T1]{fontenc}	
\usepackage[italian]{babel}

\usepackage{amsmath}
\usepackage{amsfonts}
\usepackage{amssymb}
\usepackage{graphicx}

\usepackage[left=2cm,right=2cm,top=2cm,bottom=2cm]{geometry}
\geometry{a4paper}

\usepackage{booktabs}
\usepackage{verbatim}
\usepackage{subfig}

\usepackage[cdot, thickqspace, squaren]{SIunits}
\usepackage{float}
% macro
\def\code#1{\texttt{#1}}

\title{Esperienza di Ottica 1}
\author{Gruppo BL \\ Candido Alessandro, Luzio Andrea, Mazziotti Fabrizio}

\begin{document}

\maketitle

\section{Scopo}
L'esperienza è divisa in due parti:
\begin{itemize}
	\item nella prima parte si vuole determinare la lunghezza d’onda di una riga spettrale emessa dal sodio (la riga gialla);
	\item nella seconda parte ci sono due obiettivi:
	\begin{itemize}
		\item il primo è di valutare la risoluzione dello strumento di misura, uno spettroscopio a reticolo, nella misura delle linee emesse da lampade spettrali;
		\item il secondo è quello di determinare la costante di Rydberg dalla misura della lunghezza d’onda delle righe di emissione dell’idrogeno.
	\end{itemize}
\end{itemize}

\section{Esperienza A: Misura della lunghezza d'onda della riga gialla del sodio}

\subsection{Strumentazione}

\begin{itemize}
	\item spettroscopio a prisma:
	\begin{itemize} %questa lista è solo un abbozzo, devo scriverla bene
		\item due telescopi;
		\item base rotante con goniometro con nonio etc. etc.;
		\item prisma con supporto.
	\end{itemize}
	\item lampada al cadmio;
	\item lampada al sodio;
	\item lente d'ingrandimento, per la lettura del nonio;
	\item torcia.
\end{itemize}

\section{Esperienza B: Misura della costante di Rydberg}

\subsection{Strumentazione}

\end{document}