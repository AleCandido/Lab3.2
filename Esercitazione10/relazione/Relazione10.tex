\documentclass[a4paper,10pt]{article}
\usepackage[utf8]{inputenc}
\usepackage[T1]{fontenc}	
\usepackage[italian]{babel}

\usepackage{amsmath}
\usepackage{amsfonts}
\usepackage{amssymb}
\usepackage{graphicx}

\usepackage[left=2cm,right=2cm,top=2cm,bottom=2cm]{geometry}
\geometry{a4paper}

\usepackage{booktabs}
\usepackage{verbatim}
\usepackage{subfig}
\usepackage[dvipsnames]{xcolor}  %colori
\usepackage[colorlinks=true, linkcolor=black, urlcolor=blue, citecolor=darkgray, filecolor=darkgray]{hyperref}   %per gli hyperlink

\usepackage[cdot, thickqspace, squaren]{SIunits}
\usepackage{float}

% macro
\def\code#1{\texttt{#1}}

\title{Esperienza di Franck Hertz}
\author{Gruppo BL \\ Candido Alessandro, Luzio Andrea, Mazziotti Fabrizio}

\begin{document}

\maketitle

%\begin{abstract}
%Lo scopo di questa esperienza è di
%\end{abstract}

\section{Scopo e Strumentazione}
Lo scopo dell'esperienza è Misurare le caratteristiche statiche e dinamiche delle porte NOT contenute nell’integrato \code{SN74LS04} (HEX Inverter).

La strumentazione è quella solitamente presente sul banco di lavoro, e inoltre si è usato:
\begin{itemize}
	\item \code{IC SN74LS04};
	\begin{itemize}
		\item Trimmer da $2~$K e $100~$K;
	\end{itemize}
	\item Arduino Nano; 
	\begin{itemize}
		\item \code{IC SN74LS244} octal buffer/driver; 
		\item Trimmer da $10~$K;
	\end{itemize}
\end{itemize}

%\subsection{Errori sistematici}
%
%\begin{itemize}
%
% \item Oscilloscopio digitale Tektronix TDS 1012. \newline
% 		Lo strumento è affetto da erorre sistematico del 3 \% sulle scale di tensione utilizzate, e di 100 ppm sulle scale di tempo utilizzate.
% \item Tester digitale Konig KDM-350CTF. \newline con errore sistematico del 0.5 \% sulle scale di tensione, 0.8\% su tutte le scale di resistenza utilizzate. Per quanto riguarda l'errore sistematico relativo alle capacità è del 4\%.
%
%\end{itemize}

\section{Caratteristiche statiche}

\subsection{Misura delle tensioni di operazione}

\begin{figure}[H]
	\centering
	\includegraphics[width=0.7\textwidth]{../grafici/NOTin.png}
	\caption{}
	\label{fig:NOTin}
\end{figure}

\subsection{Misura delle correnti e del fanout}

\paragraph{Correnti in ingresso}

Si è variata la tensione in ingresso agendo sul trimmer \code{R1}, mostrato in \figurename{~\ref{fig:NOTin}}.
	
Si è osservata una fascia di tensioni in ingresso della porta \code{NOT} compresa tra il valore minimo ottenibile con ruotando il trimmer e i primi valori per cui la corrente in ingresso si annullava.\footnote{\label{nota:I0}Scendeva sotto la sensibilità minima degli strumenti a disposizione per la misura di correnti.} Si è controllato anche cosa succedesse a tensioni più elevate, ma il valore della corrente in ingresso era stabilmente nullo.

I risultati della misura sono riportati in \figurename{~\ref{fig:Iin}}, le tensioni in ingresso sono state misurate con il tester digitale, mentre le correnti in ingresso con quello analogico.

\begin{figure}[H]
	\centering
	\includegraphics[width=0.5\textwidth]{../grafici/LS04.png}
	\caption{Schema del circuito elettrico di una singola porta appartenente a \code{IC SN74LS04}}
	\label{fig:LS04}
\end{figure}

Il verso delle correnti in input è uscente, infatti con un ingresso \code{HIGH} si hanno correnti nulle, mentre per un valore dell'ingresso corrispondente a \code{LOW} si osserva una corrente uscente dal circuito (vedi \figurename{~\ref{fig:Iin}}, dove le correnti sono però riportate in modulo).

Questo è coerente con quanto atteso dallo circuito elettrico riportato sul datasheet (\figurename{~\ref{fig:LS04}), dove è mostrato che la tensione di almentazione è connessa all'input, per cui, se lasciato flottante, quest'ultimo si trova \code{HIGH}, da cui il naturale comportamento delle correnti (si è verificato misurandolo lo stato dell'input flottante, che risultava effettivamente \code{HIGH}).

\begin{figure}[H]
	\centering
	\includegraphics[width=0.7\textwidth]{../grafici/Iin.pdf}
	\caption{Dati sperimentali relativi alle correnti in ingresso in funzione delle tensioni in ingresso}
	\label{fig:Iin}
\end{figure}

La corrente $I_{IH}$ è nulla, poiché al di sopra di $V_{in} = \unit{1.2}{\volt}$ risultava tale (confronta nota \ref{nota:I0}). La corrente $I_{IL}$ risultava pari a $\unit{210 \pm 10}{\micro\ampere}$ alla tensione $V_{in} = \unit{506 \pm 6}{\milli\volt}$, prossima a quella a cui è riportato il valore del datasheet, mentre risultava $\unit{260 \pm 10}{\micro\ampere}$ alla tensione $V_{in} = \unit{172 \pm 1}{\milli\volt}$, la minima che è stato possibile raggiungere.

La corrente $I_{IH}$, essendo nulla, è compatibile con il valore massimo da datasheet pari a $\unit{40}{\micro\ampere}$, ed anche $I_{IL}$ risultava essere compatibile, in entrambi i casi considerati sopra, con il valore $\unit{-1.6}{\milli\ampere}$ riportato dal datasheet.

\subparagraph{Fanout} La corrente rilevante per il fanout è la massima corrente assorbita dalla porta (ovviamente questo costituisce una limitazione per il numero di porte di questo tipo pilotabili da un'altra porta).
Dunque la corrente rilevante è la corrente $I_{IL}$, il cui massimo valore misurato è $I_{IL}$ = $\unit{260 \pm 10}{\micro\ampere}$.

%qui la corrente è negativa, quindi sputa anziché tirarsela, ma suppongo valga lo stesso

\paragraph{Correnti in uscita} Per misurare le correnti in uscita si è montato il circuito in \figurename{~\ref{fig:NOTout}}.

\begin{figure}[H]
	\centering
	\includegraphics[width=0.5\textwidth]{../grafici/NOTout.png}
	\caption{Schema del circuito usato per la misura delle correnti in uscita dalla porta \code{NOT}}
	\label{fig:NOTout}
\end{figure}

Si riportano di seguito i valori dei componenti usati per la realizzazione del circuito:

\begin{table}[H]
	\centering
	\begin{tabular}{c|c}
		\hline
		$R_1 = \unit{103.0 \pm 0.9}{\kilo\ohm}$ & $R_2 = \unit{99.7 \pm 1.1}{\ohm}$\\
		\hline
	\end{tabular}
\end{table}

Si è quindi connesso il terminale \code{A} a $V_{in}$ e si è posto quest'ultimo:
\begin{itemize}
	\item a \code{GND}, per misurare $I_{OH}$;
	\item a $V_{cc}$, cioè la tensione di alimentazione, per misurare $I_{OL}$.
\end{itemize}

In entrambi i casi si sono eseguite tre misure: nella prima si sono misurati i valori tipici di corrente di output, corrispondenti a valori tipici di tensione per l'uscita, nelle altre si sono misurate le correnti di output corrispondenti a valori limite di tensione per la logica TTL. \footnote{\href{https://www.allaboutcircuits.com/textbook/digital/chpt-3/logic-signal-voltage-levels/}{https://www.allaboutcircuits.com/textbook/digital/chpt-3/logic-signal-voltage-levels/}}

Si è stabilito di misurare le correnti leggendo la tensione ai capi della resistenza $R_2$, cioè la tensione presente tra \code{A} e \code{B}, con il tester digitale. Si riportano le misure effettuate nella tabella seguente:

\begin{table}[H]
	\centering
	\begin{tabular}{c|c|c|c|c|c|c|c|}
		\cline{2-7}
			& \multicolumn{2}{|c|}{Typical values} &	\multicolumn{2}{|c|}{TTL IN threshold}	& \multicolumn{2}{|c|}{TTL OUT threshold}\\
		\cline{2-7}
			& $V_{out}$ & $I_{out}$ & $V_{out}$ & $I_{out}$ & $V_{out}$ & $I_{out}$	\\
		\hline
		\multicolumn{1}{|c|}{$I_{OH}$} & $\unit{3.28 \pm 0.03}{\volt}$ & $\unit{1.555 \pm 0.019}{\milli\ampere}$ & $\unit{2.00 \pm 0.02}{\volt}$ & $\unit{16.47 \pm 0.20}{\milli\ampere}$ & $\unit{2.40 \pm 0.02}{\volt}$ &	$\unit{12.09 \pm 0.15}{\milli\ampere}$\\
		\hline
		\multicolumn{1}{|c|}{$I_{OL}$} & $\unit{248 \pm 2}{\milli\volt}$ & $\unit{2.59 \pm 0.04}{\milli\ampere}$ & $\unit{800 \pm 5}{\milli\volt}$ & $\unit{24.2 \pm 0.3}{\milli\ampere}$ & $\unit{400 \pm 3}{\milli\volt}$ & $\unit{10.27 \pm 0.13}{\milli\ampere}$\\
		\hline
	\end{tabular}
\end{table}
% mentre i valori delle soglie convenzionali sono esatti per quelle di input, sono solo approssimati per quelle di output (che convenzionalmente sono 0.5-2.7, vedi nota), che forse sono anche più rilevanti delle soglie di input (dato che stiamo guardando l'output di una porta)

Dal datasheet del componente si possono leggere i valori massimi per le correnti misurate, che corrispondono a $I_{OH} = \unit{-0.4}{\milli\ampere}$ e $I_{OL} = \unit{16}{\milli\ampere}$.
% e non torna un accidente, IOH, mentre IOL torna tranquillo tranquillo

\subparagraph{Variazioni repentine} Si è osservata la presenza di regioni di tensione di output in cui non si riusciva a fissare stabilmente il valore dell'uscita, e la tensione tendeva a portarsi ai margini della regione. Queste regioni (una nella misura di $I_{OH}$ e una in quella di $I_{OL}$) erano instabili proprio perché per variazioni piccole del carico si passava rapidamente da un margine all'altro della regione.
% e se devo dare una spiegazione del perché non la so e quindi mi fermo, amen

\section{Montaggio di Arduino}

\section{Caratteristiche dinamiche}

\subsection{Misura dei tempi di propagazione}

\subsection{Misura del tempo di salita}

\end{document}