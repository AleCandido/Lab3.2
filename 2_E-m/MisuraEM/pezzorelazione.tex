\documentclass[10pt,a4paper]{article}

\usepackage[utf8]{inputenc}
\usepackage[T1]{fontenc}	
\usepackage[italian]{babel}
\usepackage{amsmath}
\usepackage{amsfonts}
\usepackage{amssymb}
\usepackage{graphicx}

\usepackage[left=2cm,right=2cm,top=2cm,bottom=2cm]{geometry}
\geometry{a4paper}

\usepackage{booktabs} % for much better looking tables
\usepackage{verbatim}
\usepackage{subfig} % make it possible to include more than one captioned figure/table in a single 

% pacchetti che mi fanno schifo ma uso lo stesso (Bob è scemo, ma anche Ale...)
\usepackage[cdot, thickqspace, squaren]{SIunits}
% il miglior pacchetto che potessi desiderare
\usepackage{float}
% macro che mi piacciono
\def\code#1{\texttt{#1}}


\title{Esperienza di Franck Hertz}
\author{Gruppo BL \\ Candido Alessandro, Luzio Andrea, Mazziotti Fabrizio}

\begin{document}

\maketitle

\section{Elettroni in campo magnetico}
Dopo aver inserito il bulbo di vetro nel suo socket e acceso i vari alimentatori, si ha che gli atomi di He all'interno dell'ampolla emettono radiazione nel visibile in seguito agli urti con gli elettroni. Questo consente di visualizzare il pennello elettronico e misurarne l'orbita (vedere figura \figurename{~\ref{fig:}}).

% direi di non scrivere tensione Vheat = 7 o 2 V dato che l'esperienza richiedeva di farla variare da 3 a 6, cosi evitiamo il salto misterioso che si vede quando mettiamo a 2
Si è fissato i valori di $I_{Coil} = \unit{1.17}{A}$ e $V_{Acc} = \unit{230}{V}$ e si è variata la tensione $V_{Heat}$ responsabile dell'emissione degli elettroni per vedere se il raggio della traiettoria dipende da essa.
Per tensioni $V_{Heat}$ di $\unit{6-5}{V}$ non si osserva nessuna variazione dell'orbita degli elettroni. Quando invece $V_{Heat} = \unit{4-3}{V}$ si osserva che la traccia si divide e va a formare un'elica..
%e il raggio della traiettoria diminuisce? non ricordo. Come lo spieghiamo?


Ci si è posti nelle condizioni che si ritenevano ottimali (messo il panno scuro, fissata la posizione della macchina fotografica e il suo ingrandimento, per una successiva unica correzione delle immagini, illuminato la scala graduata) per scattare le foto con la macchina digitale e si è fissato $V_{Heat} = \unit{6}{V}$. In primo luogo si è mantenuta fissa $I_{Coil} = \unit{1.30 \pm 0.01}{A}$
%ci ho messo un errore di un digit perchè il valore ballonzolava un po'
 e si è variata la tensione di accelerazione per misurare il raggio di curvatura dell'orbita circolare (r). Durante la fase di presa dati si sono stimati i vari raggi con la scala graduata per verificare l'ordine di grandezza di e/m, utilizzando le formule \ref{} e \ref{}. Si è ottenuto e/m $\approx \unit{10^{11}}{C/Kg}$, in ottimo accordo con il valore atteso ($e/m_{th} = \unit{1.75*10^{11}}{C/Kg}$).
% (1.5-1.6*10^11) per fare il calcolo ho utilizzato le due formulette che io metterei all'inizio della relazione nella prima parte.

Dopo aver digitalizzato le immagini si è effettuata una misura più accurata di r e del suo errore mediante un fit analitico circolare. Poiché il pennello elettronico ha uno 'spessore' non trascurabile si è scelto di prendere coppie di punti, uno sulla circonferenza più esterna e uno su quella più interna (dove per esterna e interna si intende quelle per le quali si vedeva una sostanziale differenza di luminosità con l'alone circostante). Come punto della circonferenza si è preso la media aritmetica tra questi due punti e come errore la semidispersione. I dati ottenuti sono riportati in \tablename{~\ref{tab:intfisso}}.
%ci mettiamo anche una foto del fit?

\emph{TABELLA!}

In secondo luogo si è mantenuta fissa $V_{Acc} = \unit{252}{V}$ e si è variata la corrente $I_{coil}$ per misurare il raggio di curvatura dell'orbita circolare (r). Come prima si è fatta una stima dell'ordine di grandezza del valore di e/m che anche in questo caso risulta in accordo con il valore atteso.
% (1.8-2*10^11)
Anche in questo caso si è effettuato in fit analitico circolare con la stessa procedura descritta sopra. I dati ottenuti sono riportati in \tablename{~\ref{tab:vacfisso}}

\emph{TABELLA!}
\section{Valutazione errori sistematici}

\subsection{Errore dovuto alla geometria proiettiva}
Il piano della scala graduata è diverso dal piano dell'orbita, quindi si deve tener conto di ciò. Si è misurata la distanza della lente della fotocamera dal piano del righello illuminato presente nelle foto($D$), e anche la distanza dal piano dell'orbita degli elettroni sempre rispetto alla lente($d$).
Si riportano in \tablename(\ref{}).

D = 52.5 \pm 0.2
d = 45.0 \pm 0.4

Il fattore dovuto alla prospettiva risulta pari a $\unit{0.86 \pm 0.01}$. 
%da finire e riportare tabella con i valori nuovi ottenuti per r e i relativi valori di e/m

\paragraph{Osservazione}
Un ulteriore verifica che bisogna fare è che il piano contenente l'orbita sia parallelo al piano contenente la lente della fotocamera, altrimenti le immagini prese non risultano essere circonferenze ma ellissi. A tal proposito si sono effettuati fit analitici ellittici per quantificare la discrepanza 

\subsection{Effetto di distorsione dovuto alla rifrazione del bulbo}
%Si consiglia di prendere una foto della scala graduata (illuminata) posta dietro la seconda bobina in assenza il bulbo di vetro, per valutare successivamente l’effetto di distorsione introdotto dalla rifrazione del bulbo.

\subsection{Variazione $B_z(r)$/$B_z^{MAX}$}
%punto 3.d della scheda sarebbe?

Per meglio stimare la componente $B_z$ presente sulla traiettoria del fascio elettronico si è fatta una scansione con la sonda a effetto Hall facendola scorrere nella guida. Si sono ottenuti i dati in figura.\\

\begin{figure}[h!]
	\centering
	\includegraphics[width=0.80\textwidth]{../figure/B di r.png}
	\caption{Dati sperimentali scansione in $r$ di $B_z$ e relativo fit. Sulle ordinate si è rappresentato direttamente il valore di tensione output della sonda effetto Hall, misurato con il multimetro.Sovrapposto vi è un fit della curva teorica}
	\label{task6.3}
\end{figure}

Si è dunque tentato un fit con una curva teorica. Essa è stata stimata numericamente poiché gli autori non conoscono formule analitiche per gli integrali coinvolti (integrazione della formula di Biot-Savart nella circonferenza). Sono stati lasciati come parametri di fit il il raggio delle bobine ($R$), la distanza bobina centro ($z$), la corrente circolante moltiplicata pesata con il numero di spire e la costante paramagnetica del vuoto ($I$), un offset rispetto al punto $r=0$ ($a$). Si sono ottenuti i seguenti risultati:\\

0.0043+/-0.0019

0.16916+/-0.00018

0.184+/-0.012

0.189+/-0.01

\section{Calcolo di e/m}

%Trovare la media dei valori ottenuti per e/m con l’errore statistico

%Riportare in un grafico (Bz r)^2 in funzione di Vacc


\end{document}