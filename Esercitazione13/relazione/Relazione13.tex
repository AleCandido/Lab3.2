\documentclass[a4paper,10pt]{article}
\usepackage[utf8]{inputenc}
\usepackage[T1]{fontenc}
\usepackage{lmodern}	
\usepackage[italian]{babel}

\usepackage{amsmath}
\usepackage{amsfonts}
\usepackage{amssymb}

\usepackage{graphicx}
\usepackage[dvipsnames]{xcolor}  %colori

\usepackage[left=2cm,right=2cm,top=2cm,bottom=2cm]{geometry}
\geometry{a4paper}

\usepackage{verbatim}

\usepackage{booktabs}
\usepackage{subfig}
\usepackage{float}

\usepackage[colorlinks=true, linkcolor=black, urlcolor=blue, citecolor=darkgray, filecolor=darkgray]{hyperref}   %per gli hyperlink
\usepackage[italian, sort, noabbrev, capitalise]{cleveref}
\usepackage[bottom]{footmisc}

\usepackage[cdot, thickqspace, squaren]{SIunits}
% macro
\def\code#1{\texttt{#1}}

\title{Esercitazione 13: Macchine a Stati Finiti: semaforo}
\author{Gruppo BL \\ Candido Alessandro, Luzio Andrea, Mazziotti Fabrizio}

\begin{document}

\maketitle

\section{Scopo e Strumentazione}
Lo scopo dell'esercitazione è la progettazione ed implementazione di un circuito che gestisce un semaforo come applicazione del concetto di una macchina a stati finiti (FSM).
\newline

\noindent La strumentazione è quella solitamente presente sul banco di lavoro, e inoltre si è usato:
\begin{itemize}
	\item 2 Integrati 7474 – 2 FF di tipo D;
	\item 1 Integrato 7400 – 4 Porte NAND;
	\item 1 Integrato 7408 – 4 Porte AND;
	\item 1 Integrato 7432 – 4 Porte OR;
	\item 3 LED: Verde, Rosso, Giallo;
	\item 1 Switch 4 bit.
\end{itemize}

\section{Semaforo con circuiti integrati}

\section{Semaforo nello stato abilitato}

\section{Semaforo completo con lampeggiante}
\subsection{Tabelle di verità}
Si sono realizzate le tabelle di verità del circuito rappresentato in \cref{fig:??????}; nella \tablename{~\ref{tab:sem1}} sono rappresentate le transizioni di stato e le uscite in relazioni agli ingressi del circuito quando il semaforo è abilitato (Enable = 1).


\begin{table}[H]
	\centering
	\begin{tabular}{ccccccccccc}
\hline
$Enable$ & $Q1_n$ &	$Q2_n$ & $Q1_{n+1}$ & $Q2_{n+1}$ & $V_n$ & $G_n$ & $R_n$ & $V_{n+1}$ & $G_{n+1}$ & $R_{n+1}$ \\
\hline
1 & 1 & 1 & X & X & X & X & X & X & X & X \\
\hline
1 & 0 & 1 & 0 & 0 & 0 & 0 & 1 & 1 & 0 & 0 \\
1 & 0 & 0 & 1 & 0 & 1 & 0 & 0 & 1 & 1 & 0 \\
1 & 1 & 0 & 0 & 1 & 1 & 1 & 0 & 0 & 0 & 1 \\
\hline
	\end{tabular}
	\caption{Tabella di verità del circuito in esame quando il semaforo è abilitato. Gli stati contrassegnati con le X sono stati Don't Care.}
	\label{tab:sem1}
\end{table}


Le variabili $Q1_n$,$Q2_n$ rappresentano gli stati n-esimi dei due FF, mentre $Q1_{n+1}$,$Q2_{n+1}$ sono gli stati (n+1)-esimi dei due FF. Stessa cosa vale per gli stati $V_n$,$G_n$,$R_n$ e gli stati $V_{n+1}$,$G_{n+1}$,$R_{n+1}$ che rappresentano gli stati n-esimi e (n+1)-esimi delle uscite (rispettivamente il led Verde, Giallo, Rosso).
Gli stati contrassegnati con una 'X' sono stati don't care, che rappresentano i valori delle uscite dei led e dei FF indesiderate. Per come è stato costruito il circuito lo stato indesiderato ($Q1_n$ = $Q2_n$ = 1) porta a sostituire le X con quanto riportato in \tablename{~\ref{tab:inde}}.
  
\begin{table}[H]
	\centering
	\begin{tabular}{ccccccccccc}
\hline
$Enable$ & $Q1_n$ &	$Q2_n$ & $Q1_{n+1}$ & $Q2_{n+1}$ & $V_n$ & $G_n$ & $R_n$ & $V_{n+1}$ & $G_{n+1}$ & $R_{n+1}$ \\
\hline
1 & 1 & 1 & 0 & 1 & 0 &  1 & 1 & 0 & 0 & 1  \\
\hline
	\end{tabular}
	\caption{Tabella di verità per lo stato indesiderato quando il semaforo è abilitato.}
	\label{tab:inde}
\end{table}

Quindi come si può vedere dalle due tabelle, se si capita (all'accensione del circuito) nello stato indesiderato, il led giallo e rosso sono accesi contemporaneamente. Al colpo successivo di clock le uscite dei FF cadono negli stati in corrispondenza dei quali solo il led Rosso è acceso. Da questo momento in poi si alternano gli stati Rosso, Verde, Verde-Giallo e non si può più ricadere nello stato indesiderato.
\newline

Nella \tablename{~\ref{tab:lampe}} sono rappresentate le transizioni di stato e le uscite in relazioni agli ingressi del circuito quando il semaforo è disabilitato (Enable = 0). Le varie variabili utilizzate della tabella hanno la stessa definizione del caso del semaforo abilitato.
 
 \begin{table}[H]
	\centering
	\begin{tabular}{ccccccccccc}
\hline
Enable & $Q1_n$ & $Q2_n$ & $Q1_{n+1}$ & $Q2_{n+1}$ & $V_n$ & $G_n$ & $R_n$ & $V_{n+1}$ & $G_{n+1}$ & $R_{n+1}$ \\
\hline
0 & 1 & 1 & 0 & 1 & 0 & 1 & 0 & 0 & 0 & 0 \\
0 & 0 & 1 & 1 & 0 & 0 & 0 & 0 & 0 & 1 & 0 \\
0 & 0 & 0 & 1 & 0 & 0 & 0 & 0 & 0 & 1 & 0 \\
0 & 1 & 0 & 0 & 1 & 0 & 1 & 0 & 0 & 0 & 0 \\
\hline
	\end{tabular}
	\caption{Tabella di verità del circuito in esame quando il semaforo è disabilitato.}
	\label{tab:lampe}
\end{table}

Come si può vedere dalla tabella, qui non ci sono stati non permessi, poiché quando l'enable è 0, automaticamente il Led Rosso e Verde sono spenti per via delle due porte AND ai loro ingressi (vedere \cref{fig:???}). In ogni caso quindi il Led Giallo è acceso o spento e inverte il suo stato a ogni colpo di clock, come voluto.

%c'è altro che si deve scrivere riguardo le tabelle di verità?

\section{FSM in Software}

\end{document}