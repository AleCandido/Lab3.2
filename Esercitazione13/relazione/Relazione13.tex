\documentclass[a4paper,10pt]{article}
\usepackage[utf8]{inputenc}
\usepackage[T1]{fontenc}
\usepackage{lmodern}	
\usepackage[italian]{babel}

\usepackage{amsmath}
\usepackage{amsfonts}
\usepackage{amssymb}
\usepackage{bm}

\usepackage{graphicx}
\usepackage[dvipsnames]{xcolor}  %colori
\usepackage{pgf}
\usepackage{tikz}
\usetikzlibrary{arrows,shapes,snakes,automata,backgrounds,petri}	% Finite States Machine

\usepackage[left=2cm,right=2cm,top=2cm,bottom=2cm]{geometry}
\geometry{a4paper}

\usepackage{verbatim}

\usepackage{booktabs}
\usepackage{subfig}
\usepackage{float}

\usepackage[colorlinks=true, linkcolor=black, urlcolor=blue, citecolor=darkgray, filecolor=darkgray]{hyperref}   %per gli hyperlink
\usepackage[italian, sort, noabbrev, capitalise]{cleveref}
\usepackage[bottom]{footmisc}

\usepackage[cdot, thickqspace, squaren]{SIunits}
% macro
\def\code#1{\texttt{#1}}

\title{Esercitazione 13: Macchine a Stati Finiti: semaforo}
\author{Gruppo BL \\ Candido Alessandro, Luzio Andrea, Mazziotti Fabrizio}

\begin{document}

\maketitle

\section{Scopo e Strumentazione}
Lo scopo dell'esercitazione è la progettazione ed implementazione di un circuito che gestisce un semaforo come applicazione del concetto di una macchina a stati finiti (FSM).
\newline

\noindent La strumentazione è quella solitamente presente sul banco di lavoro, e inoltre si è usato:
\begin{itemize}
	\item 2 Integrati 7474 – 2 FF di tipo D;
	\item 1 Integrato 7400 – 4 Porte NAND;
	\item 1 Integrato 7408 – 4 Porte AND;
	\item 1 Integrato 7432 – 4 Porte OR;
	\item 3 LED: Verde, Rosso, Giallo;
	\item 1 Switch 4 bit.
\end{itemize}

\subsection{Specifiche}

Si riportano le specifiche del progetto:
\newline

\noindent Il semaforo può avere due modalità di funzionamento: “ABILITATO” o “DISABILITATO”.
\begin{itemize}
	\item 
	Nella modalità ABILITATO la sequenza degli stati, ripetuta ciclicamente, deve essere:
	
	\qquad LED Verde acceso $ \bm{\longrightarrow} $ LED Verde e Giallo acceso $ \bm{\longrightarrow} $ LED Rosso acceso
	
	\item Nella modalità DISABILITATO i led Verde e Rosso sono spenti ed il led Giallo lampeggia.
\end{itemize}
Tutti gli stati devono durare 1 impulso di clock. La modalità di funzionamento viene determinata tramite un interruttore che genera il segale di abilitazione (“E”=enable).

Si può scegliere se il segnale E sia attivo alto oppure attivo basso, ma fare attenzione ad essere consistenti nella definizione e nella analisi.

\section{Progettazione}

Si è scelto di implementare direttamente il semaforo con enable, in quanto è stato realizzato mantenendo la struttura fondamentale del caso abilitato (cioè apportando modifiche quanto più possibile leggere), la progettazione è avvenuta comunque in due fasi, che si procede ad illustrare.

\subsection{Semaforo nello stato abilitato}

In assenza di enable non ci sono ingressi, e il comportamento richiesto è la generazione in uscita di una sequenza di periodo $ 3 $, in cui ogni step duri un ciclo di clock.

Non c'è allora differenza tra una macchina di Mealy e una di Moore, e le uscite saranno funzione dello stato. Si è quindi deciso di codificare i $ 3 $ stati necessari usando $ 2 $ bit (cioè due flip flop). 

Si avrà quindi uno stato in eccesso, che non rientra nel ciclo. Per far sì che il circuito funzioni correttamente esso deve transire verso uno dei $ 3 $ stati permessi, in modo che in qualunque stato si trovi il flip flop all'accensione esso sia, già al secondo step, in uno dei tre stati permessi, e da lì continui a ciclare come stabilito.

Si riporta quindi in \cref{fig:wEN} il comportamento descritto, dove lo stato proibito è rappresentato con X.

\begin{figure}[H]
	\centering
	\input{../grafici/FSM.tex}
	\caption{Macchina a stati finiti per il caso di semaforo abilitato}
	\label{fig:wEN}
\end{figure}

Gli altri stati sono già stati indicati con le denominazioni relative alle uscite del circuito:
\begin{itemize}
	\item R è lo stato in cui l'uscita è costituita dal solo LED rosso acceso;
	\item V è lo stato in cui l'uscita è costituita dal solo LED verde acceso;
	\item VG è lo stato in cui l'uscita è costituita dal solo LED rosso spento, cioè i LED verde e giallo accesi contemporaneamente.
\end{itemize}
Lo schema riprodotto in \cref{fig:wEN} è quasi del tutto generale, a meno di un dettaglio: non è rilevante che la transizine dallo stato proibito sia X$ \rightarrow $R, ma è sufficiente qualunque stato permesso.
Quello riprodotto è quello che è stato scelto per il circuito implementato.


\subsection{Semaforo completo con lampeggiante}

\begin{figure}[H]
	\centering
	\begin{tikzpicture}[->,>=stealth',shorten >=1pt,auto,node distance=2.8cm,
thick]
\tikzstyle{every state}=[fill=none,draw=black,text=black]

\begin{scope}

\node[state]		 (A)              {R};	%$0-1$
\node[state]         (B) [below of=A] {V};	%$0-0$
\node[state]         (C) [below of=B] {VG};	%$1-0$
\node[state]         (D) [right of=B] {X};	%$1-1$

\path 	
(A) edge					node {}		(B)
(B) edge					node {} 	(C)
(C) edge [bend left]		node {} 	(A)
(D) edge [bend right=60]	node {} 	(A);

\end{scope}

\begin{scope}[xshift=9cm]

\node[state]		 (A')             	 	{R};	%$0-1$
\node[state]         (B') [below of=A'] 	{V};	%$0-0$
\node[state]         (C') [below of=B'] 	{VG};	%$1-0$
\node[state]         (D') [right of=B'] 	{X};	%$1-1$

\path 	
(A') edge					node {}		(B')
(B') edge					node {} 	(C')
(C') edge [bend left]		node {} 	(A')
(D') edge [bend right=60]	node {} 	(A');

\path[dashed,draw=black!40]	
(A') edge [bend left=70]		node {}		(C')
(B') edge [bend left=50]		node {}		(C')
(C') edge [bend left=70]		node {} 	(A')
(D') edge [bend left]			node {} 	(A');

\end{scope}

\draw [-to,thick,snake=snake,segment amplitude=.4mm,segment length=2mm,line after snake=1mm,black!60]
([xshift=1.3cm]D -| D) -- ([xshift=-2.7cm]B' -| A')
node [above=1mm,midway,text width=3cm,text centered,black]
{\Large Enable};

%\begin{pgfonlayer}{background}
%\filldraw [line width=0.15mm,join=round,white,draw=black,rounded corners]
%([xshift=-1cm,yshift=2mm]A.north -| B.west) rectangle ([xshift=2mm,yshift=-2mm]C.south -| D.east)
%([xshift=-1.7cm,yshift=2mm]A'.north -| B'.west) rectangle ([xshift=2mm,yshift=-2mm]C'.south -| D'.east);
%\end{pgfonlayer}
\end{tikzpicture}
	\caption{Macchine a stati finiti per i due casi: con e senza enable}
\end{figure}

\section{Implementazione}

\section{FSM in Software}

\end{document}