\documentclass[a4paper,10pt]{article}
\usepackage[utf8]{inputenc}
\usepackage[T1]{fontenc}
\usepackage{lmodern}	
\usepackage[italian]{babel}

\usepackage{amsmath}
\usepackage{amsfonts}
\usepackage{amssymb}

\usepackage{graphicx}
\usepackage[dvipsnames]{xcolor}  %colori

\usepackage[left=2cm,right=2cm,top=2cm,bottom=2cm]{geometry}
\geometry{a4paper}

\usepackage{verbatim}

\usepackage{booktabs}
\usepackage{subfig}
\usepackage{float}

\usepackage[colorlinks=true, linkcolor=black, urlcolor=blue, citecolor=darkgray, filecolor=darkgray]{hyperref}   %per gli hyperlink
\usepackage[italian, sort, noabbrev, capitalise]{cleveref}
\usepackage[bottom]{footmisc}

\usepackage[cdot, thickqspace, squaren]{SIunits}
% macro
\def\code#1{\texttt{#1}}

\title{Esercitazione 13: Macchine a Stati Finiti: semaforo}
\author{Gruppo BL \\ Candido Alessandro, Luzio Andrea, Mazziotti Fabrizio}

\begin{document}

\maketitle

\section{Scopo e Strumentazione}
Lo scopo dell'esercitazione è la progettazione ed implementazione di un circuito che gestisce un semaforo come applicazione del concetto di una macchina a stati finiti (FSM).
\newline

\noindent La strumentazione è quella solitamente presente sul banco di lavoro, e inoltre si è usato:
\begin{itemize}
	\item 2 Integrati 7474 – 2 FF di tipo D;
	\item 1 Integrato 7400 – 4 Porte NAND;
	\item 1 Integrato 7408 – 4 Porte AND;
	\item 1 Integrato 7432 – 4 Porte OR;
	\item 3 LED: Verde, Rosso, Giallo;
	\item 1 Switch 4 bit.
\end{itemize}

\section{Semaforo con circuiti integrati}

\section{Semaforo nello stato abilitato}

\section{Semaforo completo con lampeggiante}

\section{FSM in Software}

\end{document}