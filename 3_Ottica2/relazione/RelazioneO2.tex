\documentclass[a4paper,10pt]{article}
\usepackage[utf8]{inputenc}
\usepackage[T1]{fontenc}	
\usepackage[italian]{babel}

\usepackage{amsmath}
\usepackage{amsfonts}
\usepackage{amssymb}
\usepackage{graphicx}

\usepackage[left=2cm,right=2cm,top=2cm,bottom=2cm]{geometry}
\geometry{a4paper}

\usepackage{booktabs}
\usepackage{verbatim}
\usepackage{subfig}

\usepackage[cdot, thickqspace, squaren]{SIunits}
\usepackage{float}

% macro
\def\code#1{\texttt{#1}}

\title{Esperienza di Ottica 2}
\author{Gruppo BL \\ Candido Alessandro, Luzio Andrea, Mazziotti Fabrizio}

\begin{document}

\maketitle

\section{Scopo}
L'esperienza è divisa in due parti:
\begin{itemize}
	\item nella prima parte si vuole misurare la lunghezza d'onda di un laser ad He-Ne.
	\item nella seconda parte si vuole misurare la lunghezza d'onda della radiazione emessa da una lampada al mercurio.
\end{itemize}

\section{Esperienza A: Misura della lunghezza d'onda di un laser ad He-Ne}

\subsection{Strumentazione}

\begin{itemize}
	\item spettroscopio a prisma:
	\item Laser ad He-Ne;
	\item Schermo per visualizzare la figura di diffrazione;
	\item Specchio per deviare il raggio laser sullo schermo;
	\item Righello graduato di un calibro ventesimale;
	\item riga;
	\item torcia.
\end{itemize}

\subsection{Misura della lunghezza d'onda del laser ad He-Ne}
In questa prima parte dell'esperienza si vuole misurare la lunghezza d'onda di un laser ad He-Ne. Per fare ciò si è indirizzato il laser, attraverso uno specchio, sul righello graduato di un calibro ventesimale, utilizzando queste righe come un reticolo di diffrazione. Poiché il passo $d$ di questo reticolo( d = 1 mm) è molto più grande della lunghezza d'onda attesa ($\lambda_{att} \sim$ 650 nm), si è fatto in modo che il fascio laser incida sul righello con un angolo di circa $\pi/2$, altrimenti non si avrebbe modo di apprezzare la figura di diffrazione.
Questa figura si può osservare su un opportuno schermo vicino al banco di lavoro su cui è fissato un foglio di carta per la presa dati.
L'equazione che lega la posizione dei massimi di diffrazione al passo reticolare $d$ e all'angolo di incidenza $\theta_i$ è

\begin{equation}
d(sin\theta_i - sin\theta_d)= m\lambda
\label{reticolo}
\end{equation}

dove $\theta_d$ è l'angolo di riflessione.
Una rappresentazione schematica di ciò che accade è mostrata in \figurename{~\ref{fig:espa}.

\begin{figure}[H]
	\centering
	\includegraphics[width=0.7\textwidth]{../grafici/espa.png}
	\caption{Schema di un raggio incidente sul calibro e della figura di diffrazione che si viene a formare. $\theta_i$ è l'angolo di incidenza, D la distanza del reticolo dallo schermo, $\theta_n$ e $h_n$ sono rispettivamente il complementare dell'angolo di riflessione e la distanza dei vari massimi dalla quota del calibro sullo schermo.}
	\label{fig:espa}
\end{figure}

Per prima cosa si è trovata sullo schermo la quota del calibro, poiché tutte gli ordini di diffrazione devono essere riferiti a questa misura che rappresenta lo zero sull'asse dello schermo. Tale quota si è trovata misurando il punto intermedio tra la luce non riflessa dal calibro (cioè quella che si vede bene in assenza di esso) e l'ordine zero di diffrazione, che nella figura \figurename{~\ref{fig:espa} corrisponde alla quota $h_0$. Inoltre dalla posizione dell'ordine zero (m = 0), utilizzando la formula \ref{reticolo}, si può trovare l'angolo di incidenza $\theta_i$. Successivamente si sono presi gli altri ordini e, dalla misura della distanza di essi dallo zero dell'asse precedentemente trovato ($h_n$), si sono trovati i complementari degli angoli di riflessione utilizzando le formule

\begin{equation}
sin\theta_{d,n} = sin(\pi/2 - \theta_n)
\label{thetad}
\end{equation}
\begin{equation}
\theta_n = arctg(h_n/D)
\label{thetan}
\end{equation}

dove D è la distanza del calibro dallo schermo, misurata mediante un metro a nastro.
\paragraph{Errori}
Per quanto riguarda gli errori associati alle misure, si sono effettuati i seguenti accorgimenti:
\begin{itemize}
\item Per ogni misura effettuata con il metro a nastro per le lunghezze sullo schermo relative agli ordini di diffrazione si sono presi gli estremi delle bande luminose (quelle verticali\footnote{Nella parte superiore di ogni banda era presente un alone con luminosità decrescente che non si è considerato per l'acquisizione dati, mentre nella parte inferiore della banda questo alone non era presente. L'alone probabilmente è dovuto alla non perfetta monocromaticità del fascio laser, mentre si ritiene che la mancanza dell'alone nella parte inferiore sia dovuta al fatto che le tacche del calibro sono in rilievo},
%non lo so, ho sparato, se avete altre idee proponete
quelle orizzontali sono dovute solo alla larghezza del fascio) che contraddistinguono tale punto e si è considerata come valore la semisomma e come errore la semidifferenza tra questi punti.
%non abbiamo fatto così ma altrimenti dovrei scrivere " abbiamo preso un punto a caso al "centro" della banda e poi gli estremi", che è brutto. Comunque il risultato non cambierebbe di tanto quindi io lascierei cosi.
\item Per la distanza D dello schermo dal calibro, poiché lo spot del laser sul calibro è allungato, si sono identificate 3 misure: una tra lo schermo e la zona in cui il fascio laser aveva la maggior luminosità sul calibro e le altre due dallo schermo fino agli estremi di luminosità del laser sempre sul calibro. Si è quindi deciso di considerare la prima misura come l'effettiva distanza D, mentre le altre due sono state utilizzate per stimare l'errore associato a questa misura, in particolare si è scelto di prendere come errore su D la semidifferenza tra le due misure suddette. I valori ottenuti per queste 3 misure sono riportati in \tablename{~\ref{tab:D}}. Da questi dati si è ottenuto $D = 210 \pm 5 cm$.

\begin{table}[H]
	\centering
	\begin{tabular}{c|c|c|c}
\hline
205.0$\pm$0.1 & 210.1$\pm$0.1 & 214.7$\pm$0.1\\
\hline
	\end{tabular}
\caption{Distanze caratteristiche schermo-calibro espresse in cm. La misura centrale è quella che si è assunta come D, mentre le altre due servono per stimare l'errore.}
\label{tab:D}
\end{table}

\item Gli errori associati agli angoli si sono ottenuti, propagando attraverso le formule \ref{thetad} e \ref{thetan}, gli errori ottenuti per    la distanza dello schermo dal calibro (D) e per i vari ordini di diffrazione ($h_n$).
\end{itemize}
\paragraph{risultati analisi dati}
Si è ottenuto come angolo di incidenza $\theta_i = (88.10 \pm 0.05)\degree$, risultato molto vicino a $90\degree$, come ci si poteva aspettare in seguito alle condizioni in cui ci si è posti per l'esperienza. I vari valori delle altezza $h_n$ rispetto allo zero dello schermo e dei seni degli angoli di riflessione sono riportati in \tablename{~\ref{tab:data}.

\begin{table}[H]
	\centering
	\begin{tabular}{c|c|c|c|c}
x[cm] & dx[cm] & ordine & $\theta_d[\degree]$ & d$\theta_d[\degree]$ \\
\hline
6.95 & 0.07 & 0 & 88.0677 & 0.0007 \\
10.2 & 0.1 & 1 & 87.179 & 0.002 \\
12.6 & 0.2 & 2 & 86.514 & 0.004 \\
14.7 & 0.1 & 3 & 85.932 & 0.002 \\
16.5 & 0.2 & 4 & 85.434 & 0.004 \\
18.1 & 0.2 & 5 & 84.993 & 0.005 \\
19.6 & 0.2 & 6 & 84.579 & 0.006 \\
21.0 & 0.2 & 7 & 84.193 & 0.006 \\
22.3 & 0.2 & 8 & 83.835 & 0.006 \\
23.5 & 0.2 & 9 & 83.506 & 0.007 \\
24.7 & 0.2 & 10 & 83.176 & 0.007 \\
25.8 & 0.2 & 11 & 82.88 & 0.01 \\
26.9 & 0.3 & 12 & 82.57 & 0.01 \\
27.9 & 0.3 & 13 & 82.30 & 0.01 \\
29.0 & 0.3 & 14 & 82.00 & 0.01 \\
30.0 & 0.3 & 15 & 81.73 & 0.01 \\
30.9 & 0.3 & 16 & 81.48 & 0.01 \\
	\end{tabular}
\caption{Dati raccolti per i vari ordini di diffrazione e valore del seno degli angoli di riflessione. La variabile x con il suo errore dx rappresenta la distanza dei vari massimi di diffrazione dallo zero sull'asse dello schermo.}
\label{tab:data}
\end{table}



Riscrivendo l'equazione \ref{reticolo} come
\begin{align*}
sin\theta_d = - m(\lambda/d) + sin\theta_i
\end{align*} 
  
si è effettuato un fit lineare utilizzando come variabile dipendente $\sin\theta_d$ e come variabile indipendente $m$. Il grafico è riportato in \figurename{~\ref{fig:fita}.
Dal valore $d$ del passo reticolare del calibro si è quindi ottenuta una misura della lunghezza d'onda del laser ad He-Ne, mentre l'intercetta è proprio $\sin\theta_i$. I valori ottenuti sono riportati in \tablename{~\ref{tab:fita}}.


\begin{table}[H]
	\centering
	\begin{tabular}{c|c|c|c}
$\lambda$[nm] &$\lambda_{atteso}$ &$\theta_i[\degree]$& $\chi^2/DOF$ \\
\hline
627$\pm$10 & 650 & 88.11$\pm$0.04 & 0.1/15\\
\hline
	\end{tabular}
\caption{Risultati del fit lineare.}
\label{tab:fita}
\end{table}


\begin{figure}[H]
	\centering
	\includegraphics[width=0.7\textwidth]{../grafici/fita.png}
	\caption{Fit lineare dell'angolo di diffrazione $sin\theta_d$ in funzione dell'ordine di diffrazione m.}
	\label{fig:fita}
\end{figure}


Dai risultati del fit si può osservare come il valore di $\theta_i$ sia compatibile con la misura effettuata precedentemente e con le condizioni scelte nell'esperienza riguardo quest'angolo.
Il valore della lunghezza d'onda del laser ad He-Ne è dell'ordine del valore atteso (650 nm), compatibile entro circa 2$\sigma$.

Tuttavia si è notato come la misura della distanza schermo-calibro influisca significativamente sul risultato ottenuto per $\lambda$, nel senso che cambiando la misura scelta per D di 2-3cm (sempre all'interno degli estremi riportati in \ref{tab:D}), la lunghezza d'onda varia di $\sim 20nm$. Questo induce a pensare che effettivamente il punto a maggior luminosità sul calibro si discosti di un po' da quello considerato.
% io lo ho scritto, fatemi sapere se secondo voi ce la dobbiamo mettere questa cosa.

Il $\chi^2$ è sottostimato. Si ritiene che ciò sia dovuto al fatto che tutti gli errori considerati nell'elaborazione dati sono errori strumentali e non statistici, che quindi risultano essere inevitabilmente una sovrastima dell'errore da attribuire alla misura per effettuare il fit.

 

\section{Esperienza B: Misura della lunghezza d'onda della radiazione emessa da una lampada al mercurio mediante l'interferometro di Michelson}

\subsection{Strumentazione}

\begin{itemize}
	\item Inteferometro di Michelson
	\item spaziatore in alluminio, punta di riferimento per contare i massimi, filtro verde.
	\item Laser ad He-Ne;
	\item Lampada al Mercurio
	\item Schermo per visualizzare la figura di interferenza;
	\item riga;
	\item torcia.
\end{itemize}




\end{document}