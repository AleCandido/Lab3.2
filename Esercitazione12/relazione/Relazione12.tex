% !TEX encoding = UTF-8 Unicode
% !TEX TS-program = pdflatex
% !TEX root = Relazione12.tex
% !TEX spellcheck = it-IT

\documentclass[a4paper,10pt]{article}
\usepackage[utf8]{inputenc}
\usepackage[T1]{fontenc}	
\usepackage[italian]{babel}

\usepackage{amsmath}
\usepackage{amsfonts}
\usepackage{amssymb}

\usepackage{graphicx}
\usepackage[dvipsnames]{xcolor}  %colori

\usepackage[left=2cm,right=2cm,top=2cm,bottom=2cm]{geometry}
\geometry{a4paper}
\setlength\marginparwidth{40pt}
\setlength\marginparsep{1pt}

\usepackage{verbatim}
\usepackage{lipsum}

\usepackage{booktabs}
\usepackage{subfig}
\usepackage{float}
\usepackage{wrapfig}

\usepackage[colorlinks=true, linkcolor=black, urlcolor=blue, citecolor=darkgray, filecolor=darkgray]{hyperref}   %per gli hyperlink
\usepackage[italian, sort, noabbrev, capitalise]{cleveref}
\usepackage[bottom]{footmisc}

\usepackage[cdot, thickqspace, squaren]{SIunits}

% macro
\def\code#1{\texttt{#1}}

\title{Esercitazione 12: Flip-Flop e contatori}
\author{Gruppo BL \\ Candido Alessandro, Luzio Andrea, Mazziotti Fabrizio}

\begin{document}

\maketitle

\section{Scopo e Strumentazione}
Costruire alcuni circuiti logici sequenziali, progressivamente
più complessi.

La strumentazione è quella solitamente presente sul banco di lavoro, e inoltre si è usato:
\begin{itemize}
	\item Circuiti integrati;
	\begin{itemize}
		\item \code{SN74LS00} Quad NAND Gate (x2);
		\item \code{SN74LS93} 4-bit binary counter;
		\item \code{SN74LS74} Dual D-Latch (x2);
		\item \code{SN74LS86} Quad XOR Gate;
	\end{itemize}
	\item 1 DIP Switch a 4 interuttori;
	\item 1 pulsante (doppio contatto: 1 normalmente chiuso, 1 normalmente aperto);
	\item 4 diodi LED.
\end{itemize}




\section{Flip-Flop D-Latch}

\begin{figure}[H]
	\centering
	\includegraphics[width=0.7\textwidth]{../grafici/FlipFlopD.png}
	\caption{Schema del circuito del Flip-Flop di tipo D}
	\label{fig:FFD}
\end{figure}
Si è montato il circuito in \cref{fig:FFD} e si è verificato il funzionamento del circuito inviando in $D$ il segnale proveniente dal generatore di funzioni. Si è ripetuto il procedimento diverse volte, accendendo e spegnendo l'enable $E$ (controllato con uno switch) per verificare che, come previsto:\\

\begin{itemize}
\item Se l'enable è \code{HIGH} il segnale $D$ viene semplicemente ripetuto in $Q$.
\item Se l'enable è \code{LOW} il segnale $D$ è mantenuto stabile in $Q$.
%penso intendessi questo.
\item Il valore di $Q$ (e dunque di $Q$bar) vengono conservati una volta che il dispositivo è disabilitato.
\end{itemize}

Per supportare tali affermzioni sono state prese le seguenti immagini:


\begin{figure}[H]
	\centering
	\includegraphics[width=0.7\textwidth]{../grafici/enableUp.png}
	\caption{Q segue D se l'enable è \code{HIGH}}
	\label{fig:FFD}
\end{figure}

\begin{figure}[H]
	\centering
	\includegraphics[width=0.7\textwidth]{../grafici/EnableDown0.png}
	\caption{Q fisso su \code{HIGH} mentre l'enable è \code{LOW}}
	\label{fig:FFD}
\end{figure}

\begin{figure}[H]
	\centering
	\includegraphics[width=0.7\textwidth]{../grafici/EnableDown1.png}
	\caption{Q fisso su \code{LOW} mentre l'enable è \code{LOW}}
	\label{fig:FFD}
\end{figure}

Come è mostrato nelle figure capita sia che $Q$ si mantenga fisso nel suo valore \code{HIGH} che nel suo valore \code{LOW} in base allo stato logico dell'uscita quando l'enable viene diabilitato.\\
Dalla prima immagine è invecie chiaro che l'uscita della porta si limita a seguire l'ingresso se l'enable è abilitato.\\
Oltre alla mera verifica del funzionamento si è dunque misurato il ritardo fra l'ingresso e l'uscita, nella condizione $EN=$\code{HIGH}, e si è trovato un valore di $T_{rit}=\unit{32\pm 2 }{\nano\second}$.
\newline 
La funzione del NOT è quella di fornire il valore $D$bar al circuito a valle che, in pratica, è un latch RS con enable. 
Poiché questo latch è costruito con NAND allora il valore di $EN$ per il quale il successivo RS è abilitato è \code{HIGH}, in quanto in questa condizione le due porte NAND che implementano l'enable (U1a e U1d) non modificano $D$ e $D$bar, che vengono dati direttamente in input a R e a S. Quando invece $EN=$\code{LOW} le uscite delle due porte sono forzate a \code{LOW}, duenque il latch RS finale è in condizione di \code{HOLD}. Si nota come un latch costruito con porte NOR avrebbe gli stati logici dell'enable invertiti.



%bhe in realtà sarebbe > 16 medie ±(1 intervallo di campionamento + 100 parti per milione ×
%|lettura| + 0,4 ns) dovrebbe essere dunque 0.25 + .032+ 0.4=0.6 !}





% ho creato una cartella con documentazione del nuovo oscilloscopio... bisognarebbe riempirla e farla diventare buona...


\section{Divisori di frequenza}
Si è montato il circuito in \cref{fig:counter} che rappresenta un divisore di frequenza binario.
Il circuito è costituito da un contatore a 4 bit basato su FF JK. Le uscite di ogni FF JK sono collegate
ciascuna ad un Led rosso e a delle resistenze di limitazione di corrente.

\begin{figure}[H]
	\centering
	\includegraphics[width=0.7\textwidth]{../grafici/counterScheme.png}
	\caption{Schema del circuito del contatore a 4 bit}
	\label{fig:counter}
\end{figure}


Per il corretto funzionamento del circuito è necessario che almeno uno tra gli ingressi di Reset ($R_0 , R_1$)
sia posto a terra. Infatti quando uno tra $R_0$ e $R_1$ è a terra, all'uscita del NAND1 si ha 1 e quindi J=K=1 (K è flottante, quindi è a 1) su ogni JK FF; così le loro uscite (Q e $\bar{Q}$) vengono complementate ad ogni colpo del proprio clock e il circuito funziona.
\newline
\newline
Si è dunque inviato un clock ad una frequenza di circa 1Hz nel FF1 e si è verificato
che il circuito agisse come un contatore osservando sui led la codifica binaria dei numeri da 0 
a 15 (ci si mette a frequenza bassa per riuscire ad osservare i led che si accendono e si spengono).
A questo punto, dopo aver verificato il corretto funzionamento del circuito, ci si è messi ad
una frequenza di $(64.2 \pm 0.1)kHz$ e si sono guardate le uscite $Q_A,Q_B,Q_C,Q_D$ (le uscite Q di ogni JK FF). Il circuito è costruito in modo tale che $Q_A$ fa da clock al JK FF2, $Q_B$ al JK FF3 e $Q_C$ al JK FF4. Inoltre si può notare che $Q_A$ cambia stato ad ogni periodo del clock del JK FF1, $Q_B$ cambia stato ad ogni periodo del clock del JK FF2 ($Q_A$), ecc; quindi i periodi dei clock dei JK FF2-3-4 sono rispettivamente 2-4-8 volte il periodo del clock1 e le uscite $Q_A,Q_B,Q_C,Q_D$ cambieranno con una frequenza pari a 1/2, 1/4, 1/8, 1/16 rispetto alla frequenza del  clock1. Tutto ciò si è verificato con l'oscilloscopio e le forme d'onda in uscita da $Q_A,Q_B,Q_C,Q_D$ sono riportate in \cref{fig:divisor}, in cui risulta evidente quanto appena detto\footnote{Le unità di misura sarebbero di potenziale sulle ordinate e di tempi sulle ascisse nel nostro caso, ma in generale per l'analisi degli stati logici di un circuito non sono necessarie le unità fisiche, quindi non sono state riportate nella \cref{fig:divisor}.}.


\begin{figure}[H]
	\centering
	\includegraphics[width=0.7\textwidth]{../grafici/counter.pdf}
	\caption{Forme d'onda in uscita dai JK FF ($Q_A,Q_B,Q_C,Q_D$). Le unità di misura sugli assi sono arbitrarie.}
	\label{fig:divisor}
\end{figure}

%siete d'accordo su come ho scritto la footnote sopra?


Si è misurato il ritardo tra la transizione del clock1 e quella di $Q_A,Q_B,Q_C,Q_D$. I risultati sono riportati in \tablename{~\ref{tab:time}}.

\begin{table}[H]
	\centering
	\begin{tabular}{cccc}
\hline
$Q_A$ & $Q_B$ & $Q_C$ & $Q_D$ \\
36$\pm$2 ns & 42$\pm$2 ns & 53$\pm$4 ns & 55$\pm$4 ns \\
\hline
	\end{tabular}
	\caption{Ritardo tra la transizione del clock1 e quella di $Q_A,Q_B,Q_C,Q_D$.}
	\label{tab:time}
\end{table}

Come si può vedere dalla tabella il ritardo tra la transizione del clock1 e le varie uscite dei JK FF aumenta, in particolare il ritardo all'uscita dell'ultimo JK FF è maggiore di quello precedente e così via. Ciò è dovuto al fatto che il clock e il segnale in uscita ad un JK FF dipendono dai tempi di propagazione del segnale attraverso il JK FF precedente; questo fa si che si accumuli un ritardo sempre maggiore tra la transizione del primo clock e quella dei successivi. Inoltre si può notare che l'aumento di questo tempo di transizione è non lineare (come si può osservare dalla \cref{tab:time}); ciò è dovuto al fatto che i tempi di propagazione dei vari JK FF non sono tutti uguali, quindi non ci si aspetta un particolare andamento per questi tempi di transizione. 


\subsection{Segnale a 1/10 della frequenza di clock}
Si è costruito un circuito che resetti il contatore quando arriva a 10, in modo da avere un segnale di frequenza pari a 1/10 della frequenza del clock1.
Si è scelto di costruirlo con un reset asincrono (cioè che agisce non appena si raggiunge il conteggio 10). Si è presa una porta AND (utilizzando sempre i NAND) in cui gli ingressi sono rispettivamente l'uscita del JK FF2 e l'uscita del JK FF4 (cioè $Q_B$ e $Q_D$); si è poi collegata l'uscita dell'AND agli ingressi J di ogni JK FF \footnote{Gli ingressi di reset $R_0$ e $R_1$ non servono più se non per resettare inizialmente il contatore manualmente, perché ora è l'uscita dell'AND che fa le veci del reset.}. In questo modo l'AND da' in uscita 1 solo quando sono alte le uscite $Q_B$ e $Q_D$ e così il circuito si resetta. In particolare ciò accade quando il contatore arriva a 10,12,14,16, poiché non si è effettuata nessuna restrizione sulle altre due uscite ($Q_A$ e $Q_C$). Questo tuttavia non è un problema: se il contatore partisse da zero all'inizio si resetterebbe la prima volta che $Q_B$ = $Q_D$ = 1, quindi quando si arriva a 10, e continuerebbe così; se invece inizialmente il contatore non fosse resettato ma partisse da un numero casuale (anche maggiore di 10), comunque non si avrebbero problemi perchè non appena $Q_B$ = $Q_D$ = 1 la prima volta il circuito si resetterebbe e da quel momento contererebbe sempre fino a 10. Per confermare quanto appena detto si sono visualizzate tramite l'oscilloscopio le forme d'onda in $Q_A$ e $Q_D$ (\cref{fig:81})che hanno un periodo rispettivamente pari al doppio e otto volte il periodo del clock1.

\begin{figure}[H]
	\centering
	\includegraphics[width=0.7\textwidth]{../grafici/81.png}
	\caption{Forme d'onda in $Q_A$(a frequenza minore) e $Q_D$(a frequenza maggiore) quando si è montato il circuito per ottenere un segnale a 1/10 della frequenza di clock.}
	\label{fig:81}
\end{figure}

Come si può vedere dalla \cref{fig:81} ad un periodo in $Q_D$ corrispondono 5 periodi di $Q_A$, cioè 10 periodi del clock1, in accordo con quanto descritto sopra.

\pagebreak

\section{Shift register con D-Latch}

\begin{wrapfigure}{R}{0.4\textwidth}
	\vspace{-20pt}
	\centering
	\includegraphics[width=0.4\textwidth]{../grafici/shiftreg.png}
	\vspace{-10pt}
	\caption{Schema del circuito dello shift register}
	\label{fig:shift}
	\vspace{-40pt}
\end{wrapfigure}

Si è realizzato il circuito in \cref{fig:shift}, impiegando resistenza da $\sim \unit{330}{\ohm}$ e alimentando il tutto a $\sim \unit{5}{\volt}$.

In un primo momento si è provato a non inserire alcuna resistenza di pull-up, e si verificato che in un buon numero di casi lo shift register si comportava come atteso (e.g.: tutti i bit a 0 e un singolo 1 che "viaggia", oppure la situazione reciproca, funzionavano senza inconvenienti). 
Per fare ciò si è impostato un clock a frequenza molto bassa, in modo da realizzare le configurazioni desiderate dello shift register agendo solo sull'interruttore sul DIP Switch, cioè inviando manualmente i dati al primo Flip-Flop del registro (U$1$a in \cref{fig:shift}).

Si è dunque continuato verificando lo stato delle uscite dopo aver premuto il pulsante di preset, e questo è risultato essere \code{HIGH} (LED accesi), che è quanto atteso per il comportamento dei Flip-Flop di questo tipo (si sarebbe ottenuto l'opposto se si fosse scelto di pilotare con il pulsante gli ingressi di clear).

\section{Generatore di sequenze pseudo-casuali}

\begin{wrapfigure}{L}{0.4\textwidth}
	\vspace{-5pt}
	\centering
	\includegraphics[width=0.4\textwidth]{../grafici/randomseq.png}
	\vspace{-5pt}
	\caption{Schema del circuito del generatore di sequenze pseudo-casuali}
	\label{fig:random}
	\vspace{-5pt}
\end{wrapfigure}

Si è aggiunto allo shift register, \cref{fig:shift}, un "tap", realizzato con uno XOR che connettesse le uscite dei due bit più significativi (U$2$a e U$2$b in \cref{fig:random}) all'ingresso del bit meno significativo (U$1$a in \cref{fig:random}), sostituendo il collegamento all'interruttore realizzato per testare lo shift register.
\newline

Si è inviato un clock a bassa frequenza e si è premuto il pulsante di preset, lasciando che il circuito evolvesse a partire dalla configurazione con tutti i LED accesi (tutte le uscite \code{HIGH}).

Si è osservato che il circuito compieva un loop di $15$ stati, evitando lo stato in cui tutti i LED siano spenti (tutte le uscite \code{LOW}), che analizzando rapidamente il circuito risulta subito essere uno stato stabile, che quindi "bloccherebbe" l'evoluzione.

\paragraph{Resistenze di pull-up}A differenza del circuito precedente in un primo momento il circuito sembrava non funzionare correttamente: il ciclo precedentemente descritto era eseguito fino allo stato 9) (vedi sotto), mentre la transizione successiva lo riportava direttamente allo stato 1) (qualitativamente si notava anche una differenza nell'accensione/spegnimento dei LED rispetto alle altre transizioni).

Si è dunque reinserito l'interruttore e si sono provate diverse configurazioni, individuando il problema nella transizione di U$2$a in poche configurazioni, si è dunque provato a inserire una resistenza di pull-up da $\sim \unit{1}{\kilo\ohm}$ tra il clear di questo Flip-Flop e Vcc. \marginpar{Confermate che era il clear? Si, del terzo JK.}
Questo ha infine risolto il problema, e il circuito si è comportato come atteso.

\paragraph{Ciclo pseudo-casuale} Si riportano quindi in ordine gli stati osservati nel corso del ciclo completo:

\begin{equation*}
\begin{matrix}
1)	&		&	1	&	1	&	1	&	1	\\
2)	&		&	0	&	1	&	1	&	1	\\
3)	&		&	0	&	0	&	1	&	1	\\
4)	&		&	0	&	0	&	0	&	1	\\
5)	&		&	1	&	0	&	0	&	0	\\
6)	&		&	0	&	1	&	0	&	0	\\
7)	&		&	0	&	0	&	1	&	0	\\
8)	&		&	1	&	0	&	0	&	1	\\
\end{matrix}
\qquad \qquad
\begin{matrix}
9)	&		&	1	&	1	&	0	&	0	\\
10)	&		&	0	&	1	&	1	&	0	\\
11)	&		&	1	&	0	&	1	&	1	\\
12)	&		&	0	&	1	&	0	&	1	\\
13)	&		&	1	&	0	&	1	&	0	\\
14)	&		&	1	&	1	&	0	&	1	\\
15)	&		&	1	&	1	&	1	&	0	\\
\end{matrix}
\end{equation*}

\paragraph{Altri "tap" possibili} Si è trovato in rete che qualunque scelta si faccia per la coppia che genera il "tap", si ottiene un ciclo su tutti i $15$ stati possibili\footnote{Cioè escluso quello stabile.}:
\newline
\newline
\href{https://www.slideshare.net/azadajay/prbs-generation-using-s}{https://www.slideshare.net/azadajay/prbs-generation-using-s}
\newline
\newline
Si nota che nel documento citato si fa uso di una porta XNOR, ma, a patto di scambiare lo stato iniziale da tutti \code{LOW} a tutti \code{HIGH}, è equivalente.

\end{document}